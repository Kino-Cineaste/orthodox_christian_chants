\chapter{Akathist to the Mother of God}

\pagebreak

%%%%%%%%%%%%%%%%%%%
%% BEGIN Folio 1 %%
%%%%%%%%%%%%%%%%%%%

\Kontakion{\textgreek{Α}}{1}

\People To Thee, the Champion Leader, we Thy servants dedicate a feast of victory and of thanksgiving as ones rescued out of sufferings, O Theotokos: but as Thou art one with might which is invincible, from all dangers that can be do Thou deliver us, that we may cry to Thee: Rejoice, thou Bride Unwedded!

\Oikos{\textgreek{Β}}{1}

\Priest{An archangel was sent from Heaven to say to the Theotokos: Rejoice! And beholding Thee, O Lord, taking bodily form, he was amazed and with his bodiless voice he stood crying to Her such things as these:}

\PeopleRejoice Rejoice, Thou through whom \textbf{joy} will shine forth:

\Contd{Rejoice, Thou through whom the \textbf{curse} will cease!}

Rejoice, recall of fallen \textbf{A}dam:

\Contd{Rejoice, redemption of the \textbf{tears} of Eve!}

Rejoice, height inaccessible to \textbf{hu}man thoughts:

\Contd{Rejoice, depth undiscernible even for the eyes of \textbf{an}gels!}

Rejoice, for Thou art the \textbf{throne} of the King:

\Contd{Rejoice, for Thou bearest Him Who \textbf{bear}eth all!}

Rejoice, star that causest the \textbf{Sun} to appear:

\Contd{Rejoice, womb of the Divine Incar\textbf{na}tion!}

Rejoice, Thou through whom \textbf{creation} is renewed:

\Contd{Rejoice, Thou through whom we worship the Cre\textbf{a}tor!}

Rejoice, thou Bride Unwedded!

\Kontakion{\textgreek{Γ}}{2}

\Priest{Seeing herself to be chaste, the holy one said boldly to Gabriel: The marvel of thy speech is difficult for my soul to accept. How canst thou speak of a birth from a seedless conception? And She cried: Alleluia!}

\Oikos{\textgreek{Δ}}{2}

\Priest{Seeking to know knowledge that cannot be known, the Virgin cried to the ministering one: Tell me, how can a son be born from a chaste womb? Then he spake to Her in fear, only crying aloud thus:}

\PeopleRejoice Rejoice, initiate of God's in\textbf{eff}able will:

\Contd{Rejoice, assurance of those who pray in \textbf{si}lence!}

Rejoice, beginning of Christ's \textbf{mi}racles:

\Contd{Rejoice, crown of His \textbf{dog}mas!}

Rejoice, heavenly ladder by which \textbf{God} came down:

\Contd{Rejoice, bridge that conveyest us from earth to \textbf{Hea}ven!}

\pagebreak

\Kontak{1}

Взбра́нной Воево́де победи́тельная, я́ко изба́вльшеся от злых, благода́рственная воспису́ем Ти раби́ Твои́, Богоро́дице; но я́ко иму́щая держа́ву непобеди́мую, от вся́ких нас бед свободи́, да зове́м Ти: ра́дуйся, Неве́сто Неневе́стная.

\Ikos{1}

\Ierei А́нгел предста́тель с небесе́ по́слан бысть рещи́ Богоро́дице: ра́дуйся, и со безпло́тным гла́сом воплоща́ема Тя зря, Го́споди, ужаса́шеся и стоя́ше, зовы́й к Ней такова́я:

\KhorRaduisya Ра́дуйся, Е́юже рáдocть возси\textbf{я́}ет;

\Contd{ра́дуйся, Е́юже кля́тва из\textbf{че́з}нет.}

Ра́дуйся, па́дшаго Ада́ма воз\textbf{зва́}ние;

\Contd{ра́дуйся, слез Еви́ных избав\textbf{ле́}ние.}

Ра́дуйся, высото́ неудобовосходи́мая челове́ческими \textbf{по́}мыслы;

\Contd{ра́дуйся, глубино́ неудобозри́мая и а́нгельскима о\textbf{чи́}ма.}

Ра́дуйся, я́ко еси́ Царе́во се\textbf{да́}лище;

\Contd{ра́дуйся, я́ко но́сиши Но\textbf{ся́}щаго вся.}

Ра́дуйся, Звездо́, явля́ющая \textbf{Со́лн}це;

\Contd{ра́дуйся, утро́бо Боже́ственнаго вопло\textbf{ще́}ния.}

Ра́дуйся, Е́юже обнов\textbf{ля́}ется тварь;

\Contd{ра́дуйся, Е́юже покла\textbf{ня́ем}ся Творцу́.}

Ра́дуйся, Неве́сто Неневе́стная.

\Kontak{2}

\Ierei Видящи Святая Себе в чистоте, глаголет Гавриилу дерзостно: преславное твоего гласа неудобоприятельно души Моей является: безсеменнаго бо зачатия рождество како глаголеши, зовый: Аллилуиа.

\Ikos{2}

\Ierei Разум недоразумеваемый разумети Дева ищущи, возопи к служащему: из боку чисту, Сыну како есть родитися мощно, рцы Ми? К Нейже он рече со страхом, обаче зовый сице:

\KhorRaduisya Радуйся, совета неизреченнаго Таиннице;

\Contd{радуйся, молчания просящих веро.}

Радуйся, чудес Христовых начало;

\Contd{радуйся, велений Его главизно.}

Радуйся, лествице небесная, Еюже сниде Бог;

\Contd{радуйся, мосте, преводяй сущих от земли на небо.}

\pagebreak

%%%%%%%%%%%%%%%%%
%% END Folio 1 %%
%%%%%%%%%%%%%%%%%

%%%%%%%%%%%%%%%%%%%
%% BEGIN Folio 2 %%
%%%%%%%%%%%%%%%%%%%

Rejoice, wonder of angels \textbf{sound}ed abroad:

\Contd{Rejoice, wound of demons be\textbf{wailed} afar!}

Rejoice, Thou Who ineffably gavest \textbf{birth} to the Light:

\Contd{Rejoice, Thou Who didst reveal Thy \textbf{sec}ret to none!}

Rejoice, Thou Who surpassest the \textbf{know}\textbf{ledge} of the wise:

\Contd{Rejoice, Thou Who givest light to the minds of the \textbf{faithful!}}

Rejoice, thou Bride Unwedded!

\Kontakion{\textgreek{Ε}}{3}

\Priest{The power of the Most High then overshadowed the Virgin for conception, and showed Her fruitful womb as a sweet meadow to all who wish to reap salvation, as they sing: Alleluia!}

\Oikos{\textgreek{Ζ}}{3}

\Priest{Having received God into Her womb, the Virgin hastened to Elizabeth whose unborn babe at once recognized Her embrace, rejoiced, and with leaps of joy as songs, cried to the Theotokos:}

\PeopleRejoice Rejoice, branch of an Un\textbf{fad}ing Sprout:

\Contd{Rejoice, acquisition of Im\textbf{mor}tal Fruit!}

Rejoice, laborer that laborest for the \textbf{Lover} of mankind:

\Contd{Rejoice, Thou Who givest birth to the \textbf{Planter} of our life!}

Rejoice, cornland yielding a rich crop of \textbf{mer}cies:

\Contd{Rejoice, table bearing a wealth of for\textbf{give}ness!}

Rejoice, Thou Who makest to bloom the \textbf{garden} of delight:

\Contd{Rejoice, Thou Who preparest a \textbf{ha}ven for souls!}

Rejoice, acceptable incense of inter\textbf{ces}sion:

\Contd{Rejoice, propitiation of \textbf{all} the world!}

Rejoice, good will of God to \textbf{mor}tals:

\Contd{Rejoice, boldness of \textbf{mortals} before God!}

Rejoice, thou Bride Unwedded!

\Kontakion{\textgreek{Η}}{4}

\Priest{Having within a tempest of doubting thoughts, the chaste Joseph was troubled. For knowing Thee to have no husband, he suspected a secret union, O blameless one. But having learned that Thy conception was of the Holy Spirit, he said: Alleluia!}

\Oikos{\textgreek{Θ}}{4}

\Priest{While the angels were chanting, the shepherds heard of Christ's coming in the flesh, and having run to the Shepherd, they beheld Him as a blameless Lamb that had been pastured in Mary's womb, and singing to Her, they cried:}

\pagebreak

Радуйся, Ангелов многословущее чудо;

\Contd{радуйся, бесов многоплачевное поражение.}

Радуйся, Свет неизреченно родившая;

\Contd{радуйся, еже како, ни единаго же научившая.}

Радуйся, премудрых превосходящая разум;

\Contd{радуйся, верных озаряющая смыслы.}

Радуйся, Невесто Неневестная.

\Kontak{3}

\Ierei Сила Вышняго осени тогда к зачатию Браконеискусную, и благоплодная Тоя ложесна, яко село показа сладкое, всем хотящим жати спасение, всегда пети сице: Аллилуиа.

\Ikos{3}

\Ierei Имущи Богоприятную Дева утробу, востече ко Елисавети: младенец же оноя абие познав Сея целование, радовашеся, и играньми яко песньми вопияше к Богородице:

\KhorRaduisya Радуйся, отрасли неувядаемыя розго;

\Contd{радуйся, Плода безсмертнаго стяжание.}

Радуйся, Делателя делающая Человеколюбца;

\Contd{радуйся, Садителя жизни нашея рождшая.}

Радуйся, ниво, растящая гобзование щедрот;

\Contd{радуйся, трапезо, носящая обилие очищения.}

Радуйся, яко рай пищный процветаеши;

\Contd{радуйся, яко пристанище душам готовиши.}

Радуйся, приятное молитвы кадило;

\Contd{радуйся, всего мира очищение.}

Радуйся, Божие к смертным благоволение;

\Contd{радуйся, смертных к Богу дерзновение.}

Радуйся, Невесто Неневестная.

\Kontak{4}

\Ierei Бурю внутрь имея помышлений сумнительных, целомудренный Иосиф смятеся, к Тебе зря небрачней, и бракоокрадованную помышляя, Непорочная; уведев же Твое зачатие от Духа Свята, рече: Аллилуиа.

\Ikos{4}

\Ierei Слышаша пастырие Ангелов поющих плотское Христово пришествие, и текше яко к Пастырю видят Сего яко агнца непорочна, во чреве Мариине упасшася, Юже поюще реша:

\pagebreak

%%%%%%%%%%%%%%%%%
%% END Folio 2 %%
%%%%%%%%%%%%%%%%%

%%%%%%%%%%%%%%%%%%%
%% BEGIN Folio 3 %%
%%%%%%%%%%%%%%%%%%%

\PeopleRejoice Rejoice, Mother of the Lamb and the \textbf{She}pherd:

\Contd{Rejoice, fold of \textbf{ra}tional sheep!}

Rejoice, torment of invisible \textbf{e}nemies:

\Contd{Rejoice, opening of the gates of \textbf{Pa}radise!}

Rejoice, for the things of Heaven re\textbf{joice} with the earth:

\Contd{Rejoice, for the things of earth join chorus}

\Contd{with the \textbf{hea}vens!}

Rejoice, never-silent mouth of the Ap\textbf{o}stles:

\Contd{Rejoice, invincible courage of the \textbf{passion}-bearers!}

Rejoice, firm sup\textbf{port} of faith:

\Contd{Rejoice, radiant \textbf{to}ken of Grace!}

Rejoice, Thou through whom \textbf{hades} was stripped bare:

\Contd{Rejoice, Thou through whom we are clothed with \textbf{glo}ry!}

Rejoice, thou Bride Unwedded!

\Kontakion{\textgreek{Ι}}{5}

\Priest{Having sighted the divinely-moving star, the Magi followed its radiance; and holding it as a lamp, by it they sought a powerful King; and having reached the Unreachable One, they rejoiced, shouting to Him: Alleluia!}

\Oikos{\textgreek{Κ}}{5}

\Priest{The sons of the Chaldees saw in the hands of the Virgin Him Who with His hand made man. And knowing Him to be the Master, even though He had taken the form of a servant, they hastened to serve Him with gifts, and to cry to Her Who is blessed:}

\PeopleRejoice Rejoice, Mother of the Un\textbf{set}ting Star:

\Contd{Rejoice, dawn of the \textbf{mys}tic day!}

Rejoice, Thou Who didst extinguish the furnace of \textbf{er}ror:

\Contd{Rejoice, Thou Who didst enlighten the initiates}

\Contd{of the \textbf{Tri}nity!}

Rejoice, Thou Who didst banish from power

the inhuman \textbf{ty}rant:

\Contd{Rejoice, Thou Who didst show us Christ the Lord,}

\Contd{the \textbf{Lover} of mankind!}

Rejoice, Thou Who redeemest from pagan \textbf{wor}ship:

\Contd{Rejoice, Thou Who dost drag us from the \textbf{works} of mire!}

Rejoice, Thou Who didst quench the \textbf{wor}ship of fire:

\Contd{Rejoice, Thou Who rescuest from the flame of the \textbf{pass}ions!}

Rejoice, guide of the faithful to \textbf{chas}tity:

\Contd{Rejoice, gladness of all gene\textbf{ra}tions!}

Rejoice, thou Bride Unwedded!

\pagebreak

\KhorRaduisya Радуйся, Агнца и Пастыря Мати;

\Contd{радуйся, дворе словесных овец.}

Радуйся, невидимых врагов мучение;

\Contd{радуйся, райских дверей отверзение.}

Радуйся, яко небесная срадуются земным;

\Contd{радуйся, яко земная сликовствуют небесным.}

Радуйся, апостолов немолчная уста;

\Contd{радуйся, страстотерпцев непобедимая дерзосте.}

Радуйся, твердое веры утверждение;

\Contd{радуйся, светлое благодати познание.}

Радуйся, Еюже обнажися ад;

\Contd{радуйся, Еюже облекохомся славою.}

Радуйся, Невесто Неневестная.

\Kontak{5}

\Ierei Боготечную звезду узревше волсви, тоя последоваша зари, и яко светильник держаще ю, тою испытаху крепкаго Царя, и достигше Непостижимаго, возрадовашася, Ему вопиюще: Аллилуиа.

\Ikos{5}

\Ierei Видеша отроцы халдейстии на руку Девичу Создавшаго руками человеки, и Владыку разумевающе Его, аще и рабий прият зрак, потщашася дарми послужити Ему, и возопити Благословенней:

\KhorRaduisya Радуйся, Звезды незаходимыя Мати;

\Contd{радуйся, заре таинственнаго дне.}

Радуйся, прелести пещь угасившая;

\Contd{радуйся, Троицы таинники просвещающая.}

Радуйся, мучителя безчеловечнаго изметающая от начальства;

\Contd{радуйся, Господа Человеколюбца показавшая Христа.}

Радуйся, варварскаго избавляющая служения;

\Contd{радуйся, тимения изымающая дел.}

Радуйся, огня поклонение угасившая;

\Contd{радуйся, пламене страстей изменяющая.}

Радуйся, верных наставнице целомудрия;

\Contd{радуйся, всех родов веселие.}

Радуйся, Невесто Неневестная.

\pagebreak

%%%%%%%%%%%%%%%%%
%% END Folio 3 %%
%%%%%%%%%%%%%%%%%

%%%%%%%%%%%%%%%%%%%
%% BEGIN Folio 4 %%
%%%%%%%%%%%%%%%%%%%

\Kontakion{\textgreek{Λ}}{6}

\Priest{Having become God-bearing heralds, the Magi returned to Babylon, having fulfilled Thy prophecy; and having preached Thee to all as the Christ, they left Herod as a babbler who knew not how to sing: Alleluia!}

\Oikos{\textgreek{Μ}}{6}

\Priest{By shining in Egypt the light of truth, Thou didst dispel the darkness of falsehood; for its idols fell, O Saviour, unable to endure Thy strength; and those who were delivered from them cried to the Theotokos:}

\PeopleRejoice Rejoice, up\textbf{lift}ing of men:

\Contd{Rejoice, downfall of \textbf{de}mons!}

Rejoice, Thou who didst trample down the dominion of de\textbf{lu}sion:

\Contd{Rejoice, Thou who didst unmask the fraud of \textbf{i}dols!}

Rejoice, sea that didst drown the \textbf{Pharaoh} of the mind:

\Contd{Rejoice, rock that doth refresh those \textbf{thirst}ing for life!}

Rejoice, pillar of fire that guidest those in \textbf{dark}ness:

\Contd{Rejoice, shelter of the world \textbf{broader} than a cloud!}

Rejoice, sustenance replacing \textbf{man}na:

\Contd{Rejoice, minister of \textbf{ho}ly delight!}

Rejoice, land of \textbf{pro}mise:

\Contd{Rejoice, Thou from whom floweth milk and \textbf{ho}ney!}

Rejoice, thou Bride Unwedded!

\Kontakion{\textgreek{Ν}}{7}

\Priest{When Symeon was about to depart this age of delusion, Thou wast brought as a Babe to him, but Thou was recognized by him as perfect God also; wherefore, marveling at Thine ineffable wisdom, he cried: Alleluia!}

\Oikos{\textgreek{Ξ}}{7}

\Priest{The Creator showed us a new creation when He appeared to us who came from Him. For He sprang from a seedless womb, and kept it incorrupt as it was, that seeing the miracle we might sing to Her, crying out:}

\PeopleRejoice Rejoice, flower of incorrupti\textbf{bi}lity:

\Contd{Rejoice, crown of \textbf{continence!}}

Rejoice, Thou from whom shineth the Archetype of the Resur\textbf{rec}tion:

\Contd{Rejoice, Thou Who revealest the life of the \textbf{an}gels!}

Rejoice, tree of shining fruit, whereby the faithful are \textbf{nou}rished:

\Contd{Rejoice, tree of goodly shade by which many are \textbf{shel}tered!}

\pagebreak

\Kontak{6}

\Ierei Проповедницы богоноснии, бывше волсви, возвратишася в Вавилон, скончавше Твое пророчество, и проповедавше Тя Христа всем, оставиша Ирода яко буесловна, не ведуща пети: Аллилуиа.

\Ikos{6}

\Ierei Возсиявый во Египте просвещение истины, отгнал еси лжи тьму: идоли бо его, Спасе, не терпяще Твоея крепости, падоша, сих же избавльшиися вопияху к Богородице:

Радуйся, исправление человеков;

\Contd{радуйся, низпадение бесов.}

Радуйся, прелести державу поправшая;

\Contd{радуйся, идольскую лесть обличившая.}

Радуйся, море, потопившее фараона мысленнаго;

\Contd{радуйся, каменю, напоивший жаждущия жизни.}

Радуйся, огненный столпе, наставляяй сущия во тьме;

\Contd{радуйся, покрове миру, ширший облака.}

Радуйся, пище, манны приемнице;

\Contd{радуйся, сладости святыя служительнице.}

Радуйся, земле обетования;

\Contd{радуйся, из неяже течет мед и млеко.}

Радуйся, Невесто Неневестная.

\Kontak{7}

\Ierei Хотящу Симеону от нынешняго века преставитися прелестнаго, вдался еси яко младенец тому, но познался еси ему и Бог совершенный. Темже удивися Твоей неизреченней премудрости, зовый: Аллилуиа.

\Ikos{7}

\Ierei Новую показа тварь, явлься Зиждитель нам от Него бывшим, из безсеменныя прозяб утробы, и сохранив Ю, якоже бе, нетленну, да чудо видяще, воспоем Ю, вопиюще:

Радуйся, цвете нетления;

\Contd{радуйся, венче воздержания.}

Радуйся, воскресения образ облистающая;

\Contd{радуйся, ангельское житие являющая.}

Радуйся, древо светлоплодовитое, от негоже питаются вернии;

\Contd{радуйся, древо благосеннолиственное, имже покрываются мнози.}

\pagebreak

%%%%%%%%%%%%%%%%%
%% END Folio 4 %%
%%%%%%%%%%%%%%%%%

%%%%%%%%%%%%%%%%%%%
%% BEGIN Folio 5 %%
%%%%%%%%%%%%%%%%%%%

Rejoice, Thou that has carried in Thy womb the Redeemer of \textbf{cap}tives:

\Contd{Rejoice, Thou that gavest birth to the \textbf{Guide} \textbf{of} those astray!}

Rejoice, supplication before the \textbf{Righ}teous Judge:

\Contd{Rejoice, forgiveness of \textbf{ma}ny sins!}

Rejoice, robe of boldness for the \textbf{na}ked:

\Contd{Rejoice, love that dost vanquish \textbf{all} desire!}

Rejoice, thou Bride Unwedded!

\Kontakion{\textgreek{Ο}}{8}

\Priest{Having beheld a strange nativity, let us estrange ourselves from the world and transport our minds to Heaven; for the Most High God appeared on earth as a lowly man, because He wished to draw to the heights them that cry to Him: Alleluia!}

\Oikos{\textgreek{Π}}{8}

\Priest{Wholly present was the Inexpressible Word among those here below, yet in no way absent from those on high; for this was a divine condescension and not a change of place, and His birth was from a God-receiving Virgin Who heard these things:}

\PeopleRejoice Rejoice, container of the Uncon\textbf{tain}able God:

\Contd{Rejoice, door of solemn \textbf{mys}tery!}

Rejoice, report doubtful to unbe\textbf{lie}vers:

\Contd{Rejoice, undoubted boast of the \textbf{faith}ful!}

Rejoice, all-holy chariot of Him Who sitteth upon the \textbf{Che}rubim:

\Contd{Rejoice, all-glorious temple of Him Who is above the \textbf{Se}raphim!}

Rejoice, Thou Who hast united \textbf{op}posites:

\Contd{Rejoice, Thou Who hast joined virginity and \textbf{mo}therhood!}

Rejoice, Thou through whom trans\textbf{gression} hath been absolved:

\Contd{Rejoice, Thou through whom Paradise is \textbf{o}pened!}

Rejoice, key to the \textbf{king}dom of Christ:

\Contd{Rejoice, hope of et\textbf{er}nal good things!}

Rejoice, thou Bride Unwedded!

\Kontakion{\textgreek{Ρ}}{9}

\Priest{All the angels were amazed at the great act of Thine incarnation; for they saw the Unapproachable God as a man approachable to all, abiding with us, and hearing from all: Alleluia!}

\pagebreak

Радуйся, во чреве носящая Избавителя плененным;

\Contd{радуйся, рождшая Наставника заблудшим.}

Радуйся, Судии праведнаго умоление;

\Contd{радуйся, многих согрешений прощение.}

Радуйся, одеждо нагих дерзновения;

\Contd{радуйся, любы, всякое желание побеждающая.}

Радуйся, Невесто Неневестная.

\Kontak{8}

\Ierei Странное рождество видевше, устранимся мира, ум на небеса преложше: сего бо ради высокий Бог на земли явися смиренный человек, хотяй привлещи к высоте Тому вопиющия: Аллилуиа.

\Ikos{8}

\Ierei Весь бе в нижних и вышних никакоже отступи неописанное Слово: снизхождение бо Божественное, не прехождение же местное бысть, и рождество от Девы Богоприятныя, слышащия сия:

Радуйся, Бога невместимаго вместилище;

\Contd{радуйся, честнаго таинства двери.}

Радуйся, неверных сумнительное слышание;

\Contd{радуйся, верных известная похвало.}

Радуйся, колеснице пресвятая Сущаго на Херувимех;

\Contd{радуйся, селение преславное Сущаго на Серафимех.}

Радуйся, противная в тожде собравшая;

\Contd{радуйся, девство и рождество сочетавшая.}

Радуйся, Еюже разрешися преступление;

\Contd{радуйся, Еюже отверзеся рай.}

Радуйся, ключу Царствия Христова;

\Contd{радуйся, надеждо благ вечных.}

Радуйся, Невесто Неневестная.

\Kontak{9}

\Ierei Всякое естество ангельское удивися великому Твоего вочеловечения делу; неприступнаго бо яко Бога, зряще всем приступнаго Человека, нам убо спребывающа, слышаща же от всех: Аллилуиа.

\pagebreak

%%%%%%%%%%%%%%%%%
%% END Folio 5 %%
%%%%%%%%%%%%%%%%%

%%%%%%%%%%%%%%%%%%%
%% BEGIN Folio 6 %%
%%%%%%%%%%%%%%%%%%%

\Oikos{\textgreek{Σ}}{9}

\Priest{We see most eloquent orators mute as fish before Thee, O Theotokos; for they are at a loss to tell how Thou remainest a Virgin and could bear a child. But we, marveling at this mystery, cry out faithfully:}

\PeopleRejoice Rejoice, receptacle of the \textbf{Wis}dom of God:

\Contd{Rejoice, treasury of His \textbf{Pro}vidence!}

Rejoice, Thou Who showest phi\textbf{lo}sophers to be fools:

\Contd{Rejoice, Thou Who exposest the learned as ir\textbf{ra}tional!}

Rejoice, for the clever critics have become \textbf{foo}lish:

\Contd{Rejoice, for the writers of myths have \textbf{fa}ded away!}

Rejoice, Thou Who didst rend the webs of the A\textbf{the}nians:

\Contd{Rejoice, Thou Who didst fill the nets of the \textbf{fish}ermen!}

Rejoice, Thou Who drawest us from the depths of \textbf{ig}norance:

\Contd{Rejoice, Thou Who enlightenest many with \textbf{know}ledge!}

Rejoice, ship for those who \textbf{wish} to be saved:

\Contd{Rejoice, harbor for sailors on the \textbf{sea} of life!}

Rejoice, thou Bride Unwedded!

\Kontakion{\textgreek{Τ}}{10}

\Priest{Desiring to save the world, He that is the Creator of all came to it according to His Own promise, and He that, as God, is the Shepherd, for our sake appeared unto us as a man; for like calling unto like, as God He heareth: Alleluia!}

\Oikos{\textgreek{Υ}}{10}

\Priest{A bulwark art Thou to virgins, and to all that flee unto Thee, O Virgin Theotokos; for the Maker of Heaven and earth prepared Thee, O Most-pure one, dwelt in Thy womb, and taught all to call to Thee:}

\PeopleRejoice Rejoice, pillar of vir\textbf{gi}nity:

\Contd{Rejoice, gate of sal\textbf{va}tion!}

Rejoice, leader of mental for\textbf{ma}tion:

\Contd{Rejoice, bestower of di\textbf{vine} good!}

Rejoice, for Thou didst renew those con\textbf{ceived} in shame:

\Contd{Rejoice, for Thou gavest wisdom to those \textbf{robbed} of their minds!}

Rejoice, Thou Who didst foil the cor\textbf{rup}ter of minds:

\Contd{Rejoice, Thou Who gavest birth to the Sower of \textbf{pu}rity!}

Rejoice, bridechamber of a seedless \textbf{mar}riage:

\Contd{Rejoice, Thou Who dost wed the \textbf{faithful} to the Lord!}

Rejoice, good nourisher of \textbf{vir}gins:

\Contd{Rejoice, adorner of holy souls as for \textbf{mar}riage!}

Rejoice, thou Bride Unwedded!

\pagebreak

\Ikos{9}

\Ierei Ветия многовещанныя, яко рыбы безгласныя видим о Тебе, Богородице, недоумевают бо глаголати, еже како и Дева пребываеши, и родити возмогла еси. Мы же, таинству дивящеся, верно вопием:

Радуйся, премудрости Божия приятелище;

\Contd{радуйся, промышления Его сокровище.}

Радуйся, любомудрыя немудрыя являющая;

\Contd{радуйся, хитрословесныя безсловесныя обличающая.}

Радуйся, яко обуяша лютии взыскателе;

\Contd{радуйся, яко увядоша баснотворцы.}

Радуйся, афинейская плетения растерзающая;

\Contd{радуйся, рыбарския мрежи исполняющая.}

Радуйся, из глубины неведения извлачающая;

\Contd{радуйся, многи в разуме просвещающая.}

Радуйся, кораблю хотящих спастися;

\Contd{радуйся, пристанище житейских плаваний.}

Радуйся, Невесто Неневестная.

\Kontak{10}

\Ierei Спасти хотя мир, Иже всех Украситель, к сему самообетован прииде, и Пастырь сый, яко Бог, нас ради явися по нам человек: подобным бо подобное призвав, яко Бог слышит: Аллилуиа.

\Ikos{10}

\Ierei Стена еси девам, Богородице Дево, и всем к Тебе прибегающим: ибо небесе и земли Творец устрои Тя, Пречистая, вселься во утробе Твоей, и вся приглашати Тебе научи:

Радуйся, столпе девства;

\Contd{радуйся, дверь спасения.}

Радуйся, начальнице мысленнаго наздания;

\Contd{радуйся, подательнице Божественныя благости.}

Радуйся, Ты бо обновила еси зачатыя студно;

\Contd{радуйся, Ты бо наказала еси окраденныя умом.}

Радуйся, тлителя смыслов упраждняющая;

\Contd{радуйся, Сеятеля чистоты рождшая.}

Радуйся, чертоже безсеменнаго уневещения;

\Contd{радуйся, верных Господеви сочетавшая.}

Радуйся, добрая младопитательнице девам;

\Contd{радуйся, невестокрасительнице душ святых.}

Радуйся, Невесто Неневестная.

\pagebreak

%%%%%%%%%%%%%%%%%
%% END Folio 6 %%
%%%%%%%%%%%%%%%%%

%%%%%%%%%%%%%%%%%%%
%% BEGIN Folio 7 %%
%%%%%%%%%%%%%%%%%%%

\Kontakion{\textgreek{Φ}}{11}

\Priest{Every hymn is defeated that trieth to encompass the multitude of Thy many compassions; for if we offer to Thee, O Holy King, songs equal in number to the sand, nothing have we done worthy of that which Thou hast given us who shout to Thee: Alleluia!}

\Oikos{\textgreek{Χ}}{11}

\Priest{We behold the holy Virgin, a shining lamp appearing to those in darkness; for, kindling the Immaterial Light, She guideth all to divine knowledge, She illumineth minds with radiance, and is honored by our shouting these things:}

\PeopleRejoice Rejoice, ray of the no\textbf{e}tic Sun:

\Contd{Rejoice, radiance of the Un\textbf{set}ting Light!}

Rejoice, lightning that en\textbf{lighten}est our souls:

\Contd{Rejoice, thunder that terrifiest our \textbf{e}nemies!}

Rejoice, for Thou didst cause the refulgent \textbf{Light} to dawn:

\Contd{Rejoice, for Thou didst cause the river of many streams to \textbf{gush} forth!}

Rejoice, Thou Who paintest the \textbf{image} of the font:

\Contd{Rejoice, Thou Who blottest out the \textbf{stain} of sin!}

Rejoice, laver that washest the \textbf{con}science clean:

\Contd{Rejoice, cup that \textbf{drawest} up joy!}

Rejoice, aroma of the sweet \textbf{fra}grance of Christ:

\Contd{Rejoice, life of mystical \textbf{glad}ness!}

Rejoice, thou Bride Unwedded!

\Kontakion{\textgreek{Ψ}}{12}

\Priest{When the Absolver of all mankind desired to blot out ancient debts, of His Own will He came to dwell among those who had fallen from His Grace; and having torn up the handwriting of their sins, He heareth this from all: Alleluia!}

\Oikos{\textgreek{Ω}}{12}

\Priest{While singing to Thine Offspring, we all praise Thee as a living temple, O Theotokos; for the Lord Who holdeth all things in His hand dwelt in Thy womb, and He sanctified and glorified Thee, and taught all to cry to Thee:}

\PeopleRejoice Rejoice, tabernacle of \textbf{God} the Word:

\Contd{Rejoice, saint \textbf{greater} than the saints!}

Rejoice, ark gilded by the \textbf{Spi}rit:

\Contd{Rejoice, inexhaustible \textbf{treasu}ry of life!}

\pagebreak

\Kontak{11}

\Ierei Пение всякое побеждается, спростретися тщащееся ко множеству многих щедрот Твоих: равночисленныя бо песка песни аще приносим Ти, Царю Святый, ничтоже совершаем достойно, яже дал еси нам, Тебе вопиющим: Аллилуиа.

\Ikos{11}

\Ierei Светоприемную свещу, сущим во тьме явльшуюся, зрим Святую Деву, невещественный бо вжигающи огнь, наставляет к разуму Божественному вся, зарею ум просвещающая, званием же почитаемая, сими:

Радуйся, луче умнаго Солнца;

\Contd{радуйся, светило незаходимаго Света.}

Радуйся, молние, души просвещающая;

\Contd{радуйся, яко гром враги устрашающая.}

Радуйся, яко многосветлое возсияваеши просвещение;

\Contd{радуйся, яко многотекущую источаеши реку.}

Радуйся, купели живописующая образ;

\Contd{радуйся, греховную отъемлющая скверну.}

Радуйся, бане, омывающая совесть;

\Contd{радуйся, чаше, черплющая рaдocть.}

Радуйся, обоняние Христова благоухания;

\Contd{радуйся, животе тайнаго веселия.}

Радуйся, Невесто Неневестная.

\Kontak{12}

\Ierei Благодать дати восхотев, долгов древних, всех долгов Решитель человеком, прииде Собою ко отшедшим Того благодати, и раздрав рукописание, слышит от всех сице: Аллилуиа.

\Ikos{12}

Поюще Твое Рождество, хвалим Тя вси, яко одушевленный храм, Богородице: во Твоей бо вселився утробе содержай вся рукою Господь, освяти, прослави и научи вопити Тебе всех:

Радуйся, селение Бога и Слова;

\Contd{радуйся, святая святых большая.}

Радуйся, ковчеже, позлащенный Духом;

\Contd{радуйся, сокровище живота неистощимое.}

\pagebreak

%%%%%%%%%%%%%%%%%
%% END Folio 7 %%
%%%%%%%%%%%%%%%%%

%%%%%%%%%%%%%%%%%%%
%% BEGIN Folio 8 %%
%%%%%%%%%%%%%%%%%%%

Rejoice, precious diadem of \textbf{pi}ous kings:

\Contd{Rejoice, venerable boast of \textbf{rev}'rent priests!}

Rejoice, unshakable \textbf{fortress} of the Church:

\Contd{Rejoice, inviolable wall of the \textbf{king}dom!}

Rejoice, Thou through whom \textbf{victories} are obtained:

\Contd{Rejoice, Thou through whom foes fall \textbf{pros}trate!}

Rejoice, \textbf{healing} of my flesh:

\Contd{Rejoice, sal\textbf{vation} of my soul!}

Rejoice, thou Bride Unwedded!

{\textsc{\color{Maroon}Final Kontakion}}

\Priest{O all-praised Mother Who didst bear the Word, holiest of all the saints, accept now our offering, and deliver us from all misfortune, and rescue from the torment to come those that cry to Thee: Alleluia! Alleluia! Alleluia! \Thrice}

\People Alleluia, Alleluia, Alleluia! \emph{after each repetition}

\begin{center}
  {\color{Maroon}\emph{And again we sing Oikos 1 and Kontakion 1 }}
\end{center}

TODO Prayers after Akathist...

\pagebreak

Радуйся, честный венче людей благочестивых;

\Contd{радуйся, честная похвало иереев благоговейных.}

Радуйся, церкве непоколебимый столпе;

\Contd{радуйся, Царствия нерушимая стено.}

Радуйся, Еюже воздвижутся победы;

\Contd{радуйся, Еюже низпадают врази.}

Радуйся, тела моего врачевание;

\Contd{радуйся, души моея спасение.}

Радуйся, Невесто Неневестная.

\Kontak{13}

О, Всепетая Мати, рождшая всех святых Святейшее Слово! Нынешнее приемши приношение, от всякия избави напасти всех, и будущия изми муки, о Тебе вопиющих: Аллилуиа, aллилуиа, aллилуиа.

\begin{center}
  {\color{Maroon}\emph{(Kондак читается трижды)}}
\end{center}

\Ierei О, Пресвятая Госпоже Владычице Богородице, вышши еси всех Ангел и Архангел, и всея твари честнейши, помощнице еси обидимых, ненадеющихся надеяние, убогих заступнице, печальных утешение, алчущих кормительнице, нагих одеяние, больных исцеление, грешных спасение, христиан всех поможение и заступление. О, Всемилостивая Госпоже, Дево Богородице Владычице, милостию Твоею спаси и помилуй святейшия патриархи православныя, преосвященныя митрополиты, архиепископы и епископы и весь священнический и иноческий чин, и вся православныя христианы ризою Твоею честною защити; и умоли, Госпоже, из Тебе без семене воплотившагося Христа Бога нашего, да препояшет нас силою Своею свыше, на невидимыя и видимыя враги наша. О, Всемилостивая Госпоже Владычице Богородице! Воздвигни нас из глубины греховныя и избави нас от глада, губительства, от труса и потопа, от огня и меча, от нахождения иноплеменных и междоусобныя брани, и от напрасныя смерти, и от нападения вражия, и от тлетворных ветр, и от смертоносныя язвы, и от всякаго зла. Подаждь, Госпоже, мир и здравие рабом Твоим, всем православным христианом, и просвети им ум, и очи сердечнии, еже ко спасению; и сподоби ны, грешныя рабы Твоя, Царствия Сына Твоего, Христа Бога нашего; яко держава Его благословена и препрославлена, со Безначальным Его Отцем, и с Пресвятым, и Благим, и Животворящим Его Духом, ныне и присно, и во веки веков. Аминь.

О, Пресвятая Дево Мати Господа, Царице Небесе и земли! Вонми многоболезненному воздыханию души нашея, призри с высоты святыя Твоея на нас, с верою и любовию покланяющихся пречистому образу Твоему. Се бо грехми погружаемии и скорбьми обуреваемии, взирая на Твой образ, яко живей Ти сущей с нами, приносим смиренныя моления наша. Не имамы бо ни иныя помощи, ни инаго предстательства, ни утешения, токмо Тебе, о, Мати всех скорбящих и обремененных. Помози нам немощным, утоли скорбь нашу, настави на путь правый нас, заблуждающих, уврачуй и спаси безнадежных, даруй нам прочее время живота нашего в мире и тишине проводити, подаждь христианскую кончину, и на страшнем суде Сына Твоего явися нам милосердая Заступница, да всегда поем, величаем и славим Тя, яко благую Заступницу рода христианскаго, со всеми угодившими Богу. Аминь.
