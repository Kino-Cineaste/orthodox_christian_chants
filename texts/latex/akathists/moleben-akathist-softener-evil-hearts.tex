\documentclass[14pt,article,oneside]{memoir}

\setulmarginsandblock{3cm}{3cm}{*}
\setlrmarginsandblock{3cm}{3cm}{*}
\checkandfixthelayout

% \setlength{\parindent}{0em}

\setlength{\parskip}{1em}

\usepackage{fontspec}
\usepackage{xunicode}
\usepackage{xltxtra}
\defaultfontfeatures{Ligatures=TeX,Mapping=text-text}
% \setmainfont[Ligatures=TeX,Mapping=tex-text]{PonomarUnicode}
% \setmainfont[Ligatures=TeX,Mapping=tex-text]{Brill}
\setmainfont[Ligatures=TeX,Mapping=tex-text]{PT Serif}
\setmonofont[Scale=0.93]{Consolas}

\usepackage[dvipsnames]{xcolor}

\newcommand{\PriestRu}[0]{
  \noindent {\emph{\color{Maroon}Иерей:}}
}

\newcommand{\DeaconRu}[0]{
  \noindent {\emph{\color{Maroon}Диакон:}}
}

\newcommand{\ChoirRu}[0]{
  \noindent {\emph{\color{Maroon}Лик:}}
}

\newcommand{\ReaderRu}[0]{
  \noindent {\emph{\color{Maroon}Чтец:}}
}

\begin{document}

\title{Молебен с Акафистом \\
  Пресвятой Богородице \\
  в честь иконы ее \\
  «Умягчение злых сердец»}
\date{}

\maketitle

\DeaconRu{Благослови́, влады́ко.}

\PriestRu{Благослове́н Бог наш всегда́, ны́не и при́сно, и во ве́ки веко́в.}

\ChoirRu{Ами́нь.

  \noindent Царю́ Небе́сный,/ \\
  Уте́шителю, Ду́ше и́стины,/ \\
  И́же везде́ сый/ \\
  и вся исполня́яй,/ \\
  Сокро́вище благи́х/ \\
  и жи́зни Пода́телю,/ \\
  прииди́ и всели́ся в ны,/ \\
  и очи́сти ны от вся́кия скве́рны,/ \\
  и спаси́, Бла́же, ду́ши на́ша.

}

\ReaderRu{Святы́й Бо́же, Святы́й Кре́пкий, Святы́й Безсме́ртный, поми́луй нас. (трижды)\\
\vin Сла́ва Отцу́ и Сы́ну и Свято́му Ду́ху, и ны́не и при́сно и во ве́ки веко́в. Ами́нь.\\
\vin Пресвята́я Тро́ице, поми́луй нас; Го́споди, очи́сти грехи́ на́ша; Влады́ко, прости́ беззако́ния на́ша; Святы́й, посети́ и исцели́ не́мощи на́ша, и́мене Твое́го ра́ди.\\
\vin Го́споди, поми́луй. \emph{(Трижды)}\\
\vin Сла́ва Отцу́ и Сы́ну и Свято́му Ду́ху, и ны́не и при́сно и во ве́ки веко́в. Ами́нь.\\
\vin О́тче наш, И́же еси́ на небесе́х! Да святи́тся и́мя Твое́, да прии́дет Ца́рствие Твое́, да бу́дет во́ля Твоя́, я́ко на небеси́ и на земли́. Хлеб наш насу́щный да́ждь нам днесь; и оста́ви нам до́лги на́ша, я́коже и мы оставля́ем должнико́м на́шим; и не введи́ нас во искуше́ние, но изба́ви нас от лука́ваго.}

\PriestRu{Я́ко Твое́ е́сть Ца́рство, и си́ла, и сла́ва. Отца́, и Сы́на, и Свята́го Ду́ха, ны́не и при́сно, и во ве́ки веко́в.}

\ReaderRu{Ами́нь. Го́споди, поми́луй. \emph{(12 раз)}}

\DeaconRu{\emph{(глас 4)} Бог Госпо́дь, и яви́ся нам, благослове́н Гряды́й во И́мя Госпо́дне.}

\DeaconRu{Испове́дайтеся Го́сподеви, я́ко благ, яко в век ми́лость Его́.}

\ChoirRu{Бог Госпо́дь, и яви́ся нам,/ благослове́н Гряды́й/ во И́мя Госпо́дне.}

\DeaconRu{Обыше́дше обыдо́ша мя и и́менем Госпо́дним противля́хся им.}

\ChoirRu{Бог Госпо́дь, и яви́ся нам,/ благослове́н Гряды́й/ во И́мя Госпо́дне.}

\DeaconRu{Не умру́, но жив бу́ду и пове́м дела́ Госпо́дня.}

\ChoirRu{Бог Госпо́дь, и яви́ся нам,/ благослове́н Гряды́й/ во И́мя Госпо́дне.}

\ChoirRu{Ка́мень, Его́же небрего́ша зи́ждущии, Сей бысть во главу́ угла, от Го́спода бысть Сей и есть ди́вен во очесех на́ших.}

\ChoirRu{Бог Госпо́дь, и яви́ся нам,/ благослове́н Гряды́й/ во И́мя Госпо́дне.}

\noindent \textbf{Тропарь Божией Матери пред иконой Ее Умягчение злых сердец (Семистрельная)}

\noindent \emph{глас 5}

% \settowidth{\versewidth}{Твоим страданием и милосердием о нас умиляемся}
% \begin{verse}[\versewidth]
\noindent  Умягчи́ на́ша зла́я сердца́, Богоро́дице,/ \\
и напа́сти ненави́дящих нас угаси́,/ \\
и вся́кую тесноту́ души́ на́шея разреши́,/ \\
на Твой бо святы́й о́браз взира́юще,/ \\
Твои́м страда́нием и милосе́рдием о нас умиля́емся/ \\
и ра́ны Твоя́ лобыза́ем,/ \\
стрел же на́ших, Тя терза́ющих, ужаса́емся./ \\
Не даждь нам, Ма́ти Благосе́рдая,/ \\
в жестокосе́рдии на́шем и от жестокосе́рдия \\
\vin бли́жних поги́бнути,/ \\
Ты бо еси́ вои́стину злых серде́ц умягче́ние.
% \end{verse}

\vspace{1em}

\begin{framed}

  \begin{center}
    \textbf{Припевы} \emph{(Трижды)}
  \end{center}

\PriestRu{Пресвята́я Богоро́дице, спаси́ нас.}

\ChoirRu{Пресвята́я Богоро́дице, спаси́ нас.}

\PriestRu{Сла́ва Отцу́ и Сы́ну и Свято́му Ду́ху, и ны́не и при́сно и во ве́ки веко́в. Ами́нь.}

\ChoirRu{И ны́не и при́сно и во ве́ки веко́в. Ами́нь.

  Спаси́ от бед рабы́ Твоя́, Богоро́дице, / я́ко вси по Бо́зе к Тебе́ прибегáем, / я́ко к неруши́мей Стене́ и Предстáтельству.

  При́зри благосе́рдием, Всепе́тая Богоро́дице, / на мое́ лю́тое телесе́ озлобле́ние, / и исцели́ души́ моея́ боле́знь.
}

\end{framed}

\DeaconRu{Па́ки и па́ки ми́ром Го́споду помо́лимся.}

\ChoirRu{Го́споди, поми́луй.}

\DeaconRu{Заступи́, спаси́, поми́луй и сохрани́ на́с, Бо́же, Твое́ю благода́тию.}

\ChoirRu{Го́споди, поми́луй.}

\DeaconRu{Пресвяту́ю, Пречи́стую, Преблагослове́нную, Сла́вную Влады́чицу на́шу Богоро́дицу и Присноде́ву Мари́ю, со все́ми святы́ми помяну́вше, са́ми себе́ и дру́г дру́га, и ве́сь живо́т на́ш Христу́ Бо́гу предади́м.}

\ChoirRu{Тебе́, Го́споди.}

\PriestRu{Ты бо еси́ Царь ми́ра, и Спас душ на́ших, и Тебе́ сла́ву возсыла́ем, Отцу́, и Сы́ну, и Свято́му Ду́ху, ны́не и при́сно, и во ве́ки веко́в.}

\ChoirRu{Ами́нь.}

% \pagebreak
\newpage

\begin{center}
  \textbf{{\large Акафист}}
\end{center}

\begin{center}
  \textbf{Кондак 1}
\end{center}

Избра́нной Де́ве Мари́и, превы́сшей всех дще́рей земли́, Матери Сы́на Бо́жия, Е́юже дае́тся спасе́ние ми́ра, со умиле́нием взыва́ем: воззри́ на многоско́рбное житие́ на́ше, воспомяни́ ско́рби и боле́зни, я́же претерпе́ла еси́, я́ко на́ша земноро́дная, и сотвори́ с на́ми по милосе́рдию Твоему́, да зове́м Ти:\\
\vin Ра́дуйся, многоско́рбная Ма́ти Бо́жия, печа́ль на́шу в ра́дость претворя́ющая и сердца́ злых челове́к умягча́ющая.

\begin{center}
  \textbf{Икос 1}
\end{center}

А́нгел возвести́ па́стырем вифлее́мстим о Рождестве́ Спаси́теля ми́ра, и с ним мно́жество во́ев небе́сных восхваля́ху Бо́га, пою́ще: «Сла́ва в вы́шних Бо́гу, и на земли́ мир, в челове́цех благоволе́ние». Ты же, Ма́ти Бо́жия, не име́вши где главу́ преклони́ти, зане́ не бе ме́ста во оби́телех, родила́ Сы́на Своего́ Пе́рвенца в верте́пе, и пови́ла Его́ пелена́ми, и положи́ла Его́ в я́слех. Те́мже, скорбь се́рдца Твоего́ ве́дуще, вопие́м Ти:

\noindent Ра́дуйся, дыха́нием Свои́м Сы́на Своего́ возлю́бленнаго согре́вшая;\\ Ра́дуйся, Преве́чнаго Младе́нца пелена́ми пови́вшая.\\
Ра́дуйся, Носи́теля вселе́нныя мле́ком Свои́м пита́вшая;\\ Ра́дуйся, верте́п в не́бо обрати́вшая.\\
Ра́дуйся, престо́л херуви́мский соде́лавшаяся;\\ Ра́дуйся, в рождестве́ и по рождестве́ Де́ва пребы́вшая.\\
Ра́дуйся, многоско́рбная Ма́ти Бо́жия, печа́ль на́шу в ра́дость\\ \vin претворя́ющая и сердца́ злых челове́к умягча́ющая.

\pagebreak

\begin{center}
  \textbf{Кондак 2}
\end{center}

Ви́дяще Преве́чнаго Младе́нца пови́та лежа́ща в я́слех, прише́дше, поклони́шася Ему́ па́стыри вифлее́мстии и глаго́лаша рече́нное им а́нгелы о Отроча́ти. Мариа́м же соблюда́ше вся глаго́лы сия́ в се́рдце Свое́м. По проше́ствии же осьми́ дней обре́зан бе Иису́с по зако́ну Изра́илеву, я́ко Челове́к. Воспева́юще смире́ние и терпе́ние Твое́, Богоро́дице, пое́м Ве́чному Бо́гу: Аллилу́иа.

\begin{center}
  \textbf{Икос 2}
\end{center}

Ра́зум иму́ще в Бо́зе утвержде́нный и соблюда́юще Зако́н Госпо́день, в четыредеся́тый день, егда́ испо́лнишася дни́е очище́ния, вознесо́ста Иису́са во Иерусали́м роди́телие Его́, е́же поста́вити Его́ пред Го́сподем и да́ти за Него́ же́ртву по рече́нному в Зако́не Госпо́днем; мы же вопие́м Богоро́дице си́це:

\noindent Ра́дуйся, Созда́теля вселе́нныя во исполне́ние Зако́на\\ \vin в храм Иерусали́мский прине́сшая;\\ Ра́дуйся, та́мо ста́рцем Симео́ном ра́достно сре́тенная.\\
Ра́дуйся, Еди́на Чи́стая и Благослове́нная в жена́х;\\ Ра́дуйся, крест Твой, скорбьми́ преукраше́нный, во смире́нии не́сшая.\\
Ра́дуйся, во́лю Бо́жию николи́же преслу́шавшая;\\ Ра́дуйся, о́браз терпе́ния и смире́ния нам Собо́ю яви́вшая.\\
Ра́дуйся, сосу́де, испо́лненный благода́ти Ду́ха Свята́го;\\ Ра́дуйся, ро́ждшая Сы́на Единоро́днаго Отца́ Небе́снаго.\\
Ра́дуйся, многоско́рбная Ма́ти Бо́жия, печа́ль на́шу в ра́дость\\ \vin претворя́ющая и сердца́ злых челове́к умягча́ющая.

\begin{center}
  \textbf{Кондак 3}
\end{center}

Си́лою свы́ше укрепля́ема была́ еси́, Ма́ти Бо́жия, егда́ услы́шала еси́ словеса́ праведнаго ста́рца Симео́на: «Се, лежи́т Сей на паде́ние и на воста́ние мно́гим во Изра́или и в зна́мение пререка́емо, и Тебе́ же Само́й ду́шу про́йдет ору́жие, я́ко да откры́ются от мно́гих серде́ц помышле́ния». Оба́че скорбь ве́лия пронзе́ се́рдце Богоро́дицы, и возопи́ Сия́ Бо́гу: Аллилу́иа.

\begin{center}
  \textbf{Икос 3}
\end{center}

Име́я тща́ние погуби́ти Отроча́, посла́ И́род изби́ти вся де́ти, су́щия в Вифлее́ме и преде́лех его́, от двою́ ле́ту и нижа́йше, по вре́мени, е́же испыта́ от волхво́в. И се, повеле́нием Бо́жиим, да́нным чрез а́нгела во сне ста́рцу Ио́сифу, бежа́ все Свято́е Семе́йство во Еги́пет и пребы́сть та́мо седмь лет до сме́рти И́рода. Те́мже со умиле́нием воспое́м Ти, Богоро́дице:

\noindent Ра́дуйся, всю тесноту́ стра́нствия поне́сшая;\\ Ра́дуйся, я́ко вси и́доли падо́ша в стране́ Еги́петстей,\\ \vin не возмо́гше терпе́ти кре́пости Сы́на Твоего́.\\
Ра́дуйся, с нечести́выми язы́чники пребыва́вшая;\\ Ра́дуйся, из Еги́пта с Пе́рвенцем О́троком\\ \vin и Свои́м обру́чником в Назаре́т прише́дшая.\\
Ра́дуйся, со ста́рцем Ио́сифом древоде́лом в нищете́ жи́вшая;\\ Ра́дуйся, все вре́мя Свое́ в труде́х провожда́вшая.\\
Ра́дуйся, многоско́рбная Ма́ти Бо́жия, печа́ль на́шу в ра́дость\\ \vin претворя́ющая и сердца́ злых челове́к умягча́ющая.

\begin{center}
  \textbf{Кондак 4}
\end{center}

Бу́ря ско́рби обдержа́ Пречи́стую Ма́терь, егда́ возвраща́ющимася и́ма из Иерусали́ма не обрето́ша в пути́ О́трока Иису́са. Сего́ ра́ди возврати́стася во Иерусали́м, взыска́юще Его́. И бысть по трие́х днех, обрето́ста Его́ в це́ркви, седя́щаго посреде́ учи́телей, и послу́шающаго их, и вопроша́ющаго их. И рече́ к Нему́ Ма́ти Его́: «Ча́до, что сотвори́ на́ма та́ко? Се оте́ц Твой и Аз, боля́ще, иска́хом Тебе́.» И рече́ к ни́ма Иису́с: «Что я́ко иска́сте Мене́? Не ве́сте ли, я́ко в тех, я́же Отца́ Моего́, досто́ит быти Ми?» Ты же, Пречи́стая, соблюда́ла еси́ глаго́лы сия́ в се́рдце Свое́м, вопию́щи Бо́гу: Аллилу́иа.

\begin{center}
  \textbf{Икос 4}
\end{center}

Услы́шала еси́, Ма́ти Бо́жия, я́ко Иису́с прохожда́ше всю Галиле́ю, уча́ на со́нмищих, пропове́дуя Ева́нгелие Ца́рствия и исцеля́я всяк неду́г и вся́кую я́зю в лю́дех. И изы́де слух о Нем по всей стране́, и приведо́ша к Нему́ вся боля́щия разли́чными неду́ги и страстьми́, одержи́мыя бе́сы и разсла́бленныя, и исцели́ их. Ты же, Ма́ти Бо́жия, ве́дущи проро́чество, скорбя́щи се́рдцем, познава́ла еси́, я́ко вско́ре прии́дет час, егда́ Сын Твой принесе́т Себе́ в же́ртву за грехи́ ми́ра. Те́мже ублажа́ем Тя, многоско́рбная Ма́ти Бо́жия, вопию́ще:

\noindent Ра́дуйся, Сы́на Своего́ на служе́ние наро́ду иуде́йскому отда́вшая;\\ Ра́дуйся, ско́рбная се́рдцем, но поко́рная во́ли Божией.\\
Ра́дуйся, мир от пото́па грехо́внаго спа́сшая;\\ Ра́дуйся, главу́ дре́вняго зми́я сте́ршая.\\
Ра́дуйся, Себе́ в же́ртву Бо́гу прине́сшая;\\ Ра́дуйся, Госпо́дь с Тобо́ю, Благослове́нная.\\
Ра́дуйся, многоско́рбная Ма́ти Бо́жия, печа́ль на́шу в ра́дость\\ \vin претворя́ющая и сердца́ злых челове́к умягча́ющая.

\begin{center}
  \textbf{Кондак 5}
\end{center}

Ца́рствие Бо́жие на земли́ пропове́дуя, облича́ше Иису́с го́рдость фарисе́ев, мне́вших себе́ пра́ведники быти. Те́мже слы́шаще при́тчи Его́, разуме́ша, я́ко о них глаго́лет, и иска́ша взя́ти Его́, но убоя́шася наро́да, поне́же я́ко проро́ка Его́ име́яху; сия́ вся ве́дающи, скорбя́щи Богоро́дица о Сы́не Свое́м возлю́бленнем, страда́ше, да не убию́т Его́, оба́че вопию́ще Бо́гу: Аллилу́иа.

\pagebreak

\begin{center}
  \textbf{Икос 5}
\end{center}

Ви́девше не́ции от иуде́й воскреше́ние Ла́заря, идо́ша к фарисе́ем и реко́ша им, я́же сотвори́ Иису́с; и рече́ им Каиа́фа, архиере́й сый ле́ту тому́: «У́не есть нам, да еди́н челове́к у́мрет за лю́ди, а не весь наро́д поги́бнет». И от сего́ дне совеща́ша, да убию́т Его́. Мы же вопие́м Ти, Пречи́стая:

\noindent Ра́дуйся, Спаси́теля ми́ра ро́ждшая;\\ Ра́дуйся, спасе́ния на́шего глави́зно.\\
Ра́дуйся, от рожде́ния в Ма́терь Спа́су на́шему предызбра́нная;\\ Ра́дуйся, Ма́ти Бо́жия, обрече́нная на страда́ние.\\
Ра́дуйся, Благослове́нная, Цари́ца Небе́сная соде́лавшаяся;\\ Ра́дуйся, вы́ну за ны моля́щаяся.\\
Ра́дуйся, многоско́рбная Ма́ти Бо́жия, печа́ль на́шу в ра́дость\\ \vin претворя́ющая и сердца́ злых челове́к умягча́ющая.

\begin{center}
  \textbf{Кондак 6}
\end{center}

Пре́жде пропове́дник сло́ва Бо́жия, последи́ же преда́тель, Иу́да Искарио́тский, еди́н от обоюна́десяте апо́столов, и́де ко архиере́ом, преда́ти хотя́ Учи́теля своего́; они́ же, слы́шавше, обра́довашася и обеща́ша ему́ сре́бренники да́ти. Ты же, Богома́ти, скорбя́щи о Сы́не Свое́м возлю́бленнем, вопия́ла еси́ го́рце Бо́гу: Аллилу́иа.

\begin{center}
  \textbf{Икос 6}
\end{center}

Возсия́ла есть после́дняя ве́черя ученико́м Христо́вым, на нейже Учи́тель умы́ но́зе им, о́браз смире́ния показу́я, и рече́ им: «Еди́н от вас преда́ст Мя, яды́й со Мно́ю». Мы же, го́рю Матери Божией сострада́юще, воспева́ем Ей:

\noindent Ра́дуйся, Ма́ти Бо́жия, му́кою серде́чною истомле́нная;\\ Ра́дуйся, вся претерпе́вшая в юдо́ли сей многоско́рбней.\\
Ра́дуйся, в моли́тве успокое́ние находи́вшая;\\ Ра́дуйся, всех скорбя́щих Ра́досте.\\
Ра́дуйся, утоле́ние на́ших печа́лей;\\ Ра́дуйся, из ти́ны грехо́вныя нас спаса́ющая.\\
Ра́дуйся, многоско́рбная Ма́ти Бо́жия, печа́ль на́шу в ра́дость\\ \vin претворя́ющая и сердца́ злых челове́к умягча́ющая.

\begin{center}
  \textbf{Кондак 7}
\end{center}

Хотя́й яви́ти любо́вь Свою́ к ро́ду челове́ческому Госпо́дь наш Иису́с Христо́с на Та́йней ве́чери, благослови́в и преломи́в хлеб, даде́ Свои́м ученико́м и апо́столом, рек: «Приими́те, яди́те, сие́ есть те́ло Мое́» и, прие́м ча́шу и хвалу́ возда́в, даде́ им, глаго́ля: «Сия́ есть кровь Моя́ Но́ваго Заве́та, я́же за мно́гия излива́емая во оставле́ние грехо́в». Благодаря́ще Милостиваго Бо́га за милосе́рдие Его́ к нам неизрече́нное, пое́м Ему́: Аллилу́иа.

\begin{center}
  \textbf{Икос 7}
\end{center}

Но́вое зна́мение ми́лости Своея́ яви́ Госпо́дь ученико́м Свои́м, обеща́я им посла́ти ина́го Уте́шителя — Ду́ха и́стины, И́же от Отца́ исхо́дит, да свиде́тельствует о Нем. Тебе́ же, Матери Божией, па́ки освяще́нней Ду́хом Святы́м в день Пятьдеся́тницы, вопие́м си́це:

\noindent Ра́дуйся, Ду́ха Свята́го обита́лище;\\ Ра́дуйся, черто́же всесвяты́й.\\
Ра́дуйся, простра́нное селе́ние Бо́га Сло́ва;\\ Ра́дуйся, Би́сер Боже́ственный произве́дшая.\\
Ра́дуйся, рождество́м Твои́м ра́йския две́ри нам отверза́ющая;\\ Ра́дуйся, зна́мение ми́лости Божией нам Собо́ю яви́вшая.\\
Ра́дуйся, многоско́рбная Ма́ти Бо́жия, печа́ль на́шу в ра́дость\\ \vin претворя́ющая и сердца́ злых челове́к умягча́ющая.

\pagebreak

\begin{center}
  \textbf{Кондак 8}
\end{center}

Стра́нно и приско́рбно нам слы́шати, я́ко лобза́нием предаде́ Учи́теля своего́ и Го́спода Иу́да Искарио́тский. Спи́ра же и ты́сящник и слуги́ архиере́йстии взя́ша Иису́са и связа́ша Его́, и ведо́ша к первосвяще́ннику А́нне пе́рвее, посе́м же к Каиа́фе архиере́ови. Ма́терь же Бо́жия, сме́ртнаго сове́та на Сы́на Своего́ возлю́бленнаго ожида́ющи, вопия́ше Бо́гу: Аллилу́иа.

\begin{center}
  \textbf{Икос 8}
\end{center}

Вси иуде́и ведо́ша Иису́са от Каиа́фы в прето́р к Пила́ту, глаго́люще Его́ злоде́я быти. Пила́т же вопроси́ Его́ и глаго́ла им, я́ко ни еди́ныя вины́ обре́те в Нем. Мы же Матери Божией, поноше́ния Сы́на Своего́ ви́дящей, уми́льно вопие́м:

\noindent Ра́дуйся, се́рдце го́рем истерза́нное име́вшая;\\ Ра́дуйся, сле́зы о Сы́не Свое́м пролива́вшая.\\
Ра́дуйся, Ча́до Свое́ люби́мое на суди́лище преда́нное зре́вшая;\\ Ра́дуйся, вся претерпе́вшая без ро́пота, я́ко раба́ Госпо́дня.\\
Ра́дуйся, стеня́щая и рыда́ющая;\\ Ра́дуйся, Цари́це небесе́ и земли́, прие́млющая моли́твы раб Свои́х.\\
Ра́дуйся, многоско́рбная Ма́ти Бо́жия, печа́ль на́шу в ра́дость\\ \vin претворя́ющая и сердца́ злых челове́к умягча́ющая.

\begin{center}
  \textbf{Кондак 9}
\end{center}

Вси ро́ди блажа́т Тя, честне́йшую Херуви́м и сла́внейшую без сравне́ния Серафи́м, Влады́чицу и Ма́терь Изба́вителя на́шего, рождество́м Свои́м ра́дость всему́ ми́ру прине́сшую, последи́ же скорбь ве́лию име́вшую, ви́дящи Сы́на Своего́ возлю́бленнаго на поруга́ние, бие́ние и смерть пре́даннаго. Мы же умиле́нное пе́ние прино́сим Ти, Пречи́стая, вопию́ще Всемогу́щему Бо́гу: Аллилу́иа.

\pagebreak

\begin{center}
  \textbf{Икос 9}
\end{center}

Вити́и многовеща́ннии не возмо́гут изрещи́ всех страда́ний, Тобо́ю пренесе́нных, Спа́се наш, егда́ во́ини, спле́тше вене́ц из те́рния и возложи́вше на главу́ Твою́, и в ри́зу багря́ну обле́кше Тя, глаго́лаху: «Ра́дуйся, Царю́ Иуде́йский», и бия́ху Тя по лани́тома. Мы же, Ма́ти Бо́жия, страда́ния Твоя́ познава́юще, вопие́м Ти:

\noindent Ра́дуйся, Сы́на Своего́ бие́ннаго зре́вшая;\\ Ра́дуйся, багряни́цею и терно́вым венце́м оде́яннаго зре́вшая.\\
Ра́дуйся, Его́же мле́ком Свои́м пита́ла еси́, терза́емаго ви́девшая;\\ Ра́дуйся, страда́нием Его́ спострада́вшая.\\
Ра́дуйся, все́ми ученики́ Его́ оста́вленнаго ви́девшая;\\ Ра́дуйся, непра́ведными судия́ми Его́ осужде́ннаго зре́вшая.\\
Ра́дуйся, многоско́рбная Ма́ти Бо́жия, печа́ль на́шу в ра́дость\\ \vin претворя́ющая и сердца́ злых челове́к умягча́ющая.

\begin{center}
  \textbf{Кондак 10}
\end{center}

Спасти́ хотя́ Иису́са, рече́ Пила́т иуде́ем: «Есть же обы́чай вам, да еди́наго вам отпущу́ на Па́сху. Хо́щете ли у́бо, да отпущу́ вам Царя́ Иуде́йска?» Но вси возопи́ша, глаго́люще: «Не Сего́, но Вара́вву». Прославля́юще милосе́рдие Отца́ Небе́снаго, И́же та́ко возлюби́ мир, я́ко Сы́на Своего́ Единоро́днаго отда́л есть на кре́стную смерть, да искупи́т ны от ве́чныя сме́рти, вопие́м Ему́: Аллилу́иа.

\begin{center}
  \textbf{Икос 10}
\end{center}

Стена́ и огражде́ние бу́ди нам, Влады́чице, изнемога́ющим от ско́рби и боле́зни. Ты бо и Сама́ страда́ла еси́, слы́шащи иуде́ев, вопию́щих: «Распни́, распни́ Его́!» Ны́не же услы́ши нас, вопию́щих Ти:

\noindent Ра́дуйся, Ма́ти милосе́рдия, отъе́млющая вся́кую слезу́\\ \vin от лю́те стра́ждущих;\\ Ра́дуйся, слезу́ умиле́ния нам подаю́щая.\\
Ра́дуйся, погиба́ющих гре́шников спаса́ющая;\\ Ра́дуйся, предста́тельство христиа́н непосты́дное.\\
Ра́дуйся, от страсте́й нас избавля́ющая;\\ Ра́дуйся, се́рдцу сокруше́нному отра́ду подаю́щая.\\
Ра́дуйся, многоско́рбная Ма́ти Бо́жия, печа́ль на́шу в ра́дость\\ \vin претворя́ющая и сердца́ злых челове́к умягча́ющая.

\begin{center}
  \textbf{Кондак 11}
\end{center}

Пе́ние всеумиле́нное Спаси́телю ми́ра прино́сим, на во́льное страда́ние ше́дшему и крест Свой не́сшему. Во́ини же, прише́дше на Голго́фу, пропя́ша Его́. Стоя́ху при кресте́ Иису́сове Ма́ти Его́, и сестра́ Ма́тере Его́ — Мари́я Клео́пова, и Мари́я Магдали́на. Иису́с же, ви́дев Ма́терь и ученика́ стоя́ща, его́же любля́ше, глаго́ла Матери Свое́й: «Же́но, се сын Твой»; пото́м глаго́ла ученику́: «Се Ма́ти твоя́». И от того́ ча́са поя́т Ю учени́к во своя́си. Ты же, Ма́ти Бо́жия, обезче́щенна ви́дящи на кресте́ Сы́на и Го́спода, терза́лася еси́, вопию́щи Бо́гу: Аллилу́иа.

\begin{center}
  \textbf{Икос 11}
\end{center}

Сы́не Мой и Бо́же Предве́чный, Тво́рче всех тва́рей, Го́споди! Ка́ко те́рпиши на кресте́ страсть, — Чи́стая Де́ва, пла́чущи, глаго́лаше. — О стра́шнем Твое́м Рождестве́, Сы́не Мой, па́че всех матере́й возвели́чена бых, но увы́ Мне: ны́не Тя ви́дящи, распаля́юся утро́бою». Мы же, сле́зы пролива́юще и Тебе́ вне́млюще, вопие́м Ти:

\noindent Ра́дуйся, ра́дости и весе́лия лише́нная;\\ Ра́дуйся, во́льная страда́ния Сы́на Твоего́ на кресте́ ви́девшая.\\
Ра́дуйся, Ча́до Свое́ возлю́бленное уязвле́нным зре́вшая;\\ Ра́дуйся, А́гнца, на закла́ние ведо́маго, Ма́ти.\\
Ра́дуйся, Изба́вителя язв душе́вных и теле́сных\\ \vin я́звами покры́таго ви́девшая;\\ Ра́дуйся, Сы́на Своего́ воскре́сшаго из ме́ртвых зре́вшая.\\
Ра́дуйся, многоско́рбная Ма́ти Бо́жия, печа́ль на́шу в ра́дость\\ \vin претворя́ющая и сердца́ злых челове́к умягча́ющая.

\begin{center}
  \textbf{Кондак 12}
\end{center}

Благода́ть пода́ждь нам, Спа́се Всеми́лостивый, на Кресте́ дух Свой испусти́вый и согреше́ний на́ших рукописа́ние растерза́вый. «Се, све́те Мой сла́дкий, наде́ждо и животе́ Мой благи́й, Бо́же Мой, уга́сл еси́ на Кресте́», — Де́ва, стеня́щи, глаго́лаше. Потщи́ся, Ио́сифе, к Пила́ту приступи́ти и проси́ его́ сня́ти с дре́ва Учи́теля своего́ уязвле́ннаго: «Даждь ми Сего́ стра́ннаго, И́же не и́мать где главы́ преклони́ти». Богома́терь же, ви́девши Сы́на Своего́ безсла́вна, на́га на дре́ве, возопи́: «Увы́ Мне, Ча́до Мое́, увы́ Мне, све́те Мой, ду́шу про́йде Мне ору́жие по рече́нию праведнаго ста́рца Симео́на». Мы же, Пречи́стей Де́ве состра́ждуще, вопие́м Бо́гу: Аллилу́иа.

\begin{center}
  \textbf{Икос 12}
\end{center}

Пою́ще милосе́рдие Твое́, Человеколю́бче, покланя́емся бога́тству ми́лости Твоея́, Влады́ко. «Созда́ние Твое́ спасти́ хотя́, смерть подъя́л еси́, — рече́ Пречи́стая, — но Воскресе́нием Твои́м, Спа́се, поми́луй всех нас». Мы же Богоро́дице, о нас моля́щейся, вопие́м:

\noindent Ра́дуйся, ме́ртваго зря́щая, бездыха́нна Преблага́го Го́спода;\\ Ра́дуйся, лобза́вшая те́ло Сы́на Твоего́ возлю́бленнаго.\\
Ра́дуйся, Све́та Своего́ сла́дкаго на́га и уязвле́ннаго Мертвеца́ зре́вшая;\\ Ра́дуйся, Сы́на Своего́ гро́бу преда́вшая.\\
Ра́дуйся, плащани́цею но́вою те́ло Его́ обви́вшая;\\ Ра́дуйся, Воскресе́ние Его́ ви́девшая.\\
Ра́дуйся, многоско́рбная Ма́ти Бо́жия, печа́ль на́шу в ра́дость\\ \vin претворя́ющая и сердца́ злых челове́к умягча́ющая.

\begin{center}
  \textbf{Кондак 13}
\end{center}

О Всепе́тая Ма́ти, изнемога́ющая от ско́рби у креста́ Сы́на Твоего́ и Бо́га! Внемли́ воздыха́нием и слеза́м на́шим, умягчи́ зла́я сердца́, востаю́щая на нас, изба́ви от ско́рби, боле́зней и ве́чныя сме́рти всех упова́ющих на неизрече́нное Твое́ милосе́рдие и вопию́щих Бо́гу: Аллилу́иа.

\emph{Этот кондак читается трижды, зате́м 1-й икос «А́нгел возвести́ …» и 1-й кондак «Избра́нной Де́ве Мари́и …»:}

\begin{center}
  \textbf{Икос 1}
\end{center}

А́нгел возвести́ па́стырем вифлее́мстим о Рождестве́ Спаси́теля ми́ра, и с ним мно́жество во́ев небе́сных восхваля́ху Бо́га, пою́ще: «Сла́ва в вы́шних Бо́гу, и на земли́ мир, в челове́цех благоволе́ние». Ты же, Ма́ти Бо́жия, не име́вши где главу́ преклони́ти, зане́ не бе ме́ста во оби́телех, родила́ Сы́на Своего́ Пе́рвенца в верте́пе, и пови́ла Его́ пелена́ми, и положи́ла Его́ в я́слех. Те́мже, скорбь се́рдца Твоего́ ве́дуще, вопие́м Ти:

\noindent Ра́дуйся, дыха́нием Свои́м Сы́на Своего́ возлю́бленнаго согре́вшая;\\ Ра́дуйся, Преве́чнаго Младе́нца пелена́ми пови́вшая.\\
Ра́дуйся, Носи́теля вселе́нныя мле́ком Свои́м пита́вшая;\\ Ра́дуйся, верте́п в не́бо обрати́вшая.\\
Ра́дуйся, престо́л херуви́мский соде́лавшаяся;\\ Ра́дуйся, в рождестве́ и по рождестве́ Де́ва пребы́вшая.\\
Ра́дуйся, многоско́рбная Ма́ти Бо́жия, печа́ль на́шу в ра́дость\\ \vin претворя́ющая и сердца́ злых челове́к умягча́ющая.

\pagebreak

\begin{center}
  \textbf{Кондак 1}
\end{center}

Избра́нной Де́ве Мари́и, превы́сшей всех дще́рей земли́, Матери Сы́на Бо́жия, Е́юже дае́тся спасе́ние ми́ра, со умиле́нием взыва́ем: воззри́ на многоско́рбное житие́ на́ше, воспомяни́ ско́рби и боле́зни, я́же претерпе́ла еси́, я́ко на́ша земноро́дная, и сотвори́ с на́ми по милосе́рдию Твоему́, да зове́м Ти:\\
\vin Ра́дуйся, многоско́рбная Ма́ти Бо́жия, печа́ль на́шу в ра́дость претворя́ющая и сердца́ злых челове́к умягча́ющая.

\begin{center}
\emph{Конец Акафиста}
\end{center}


\DeaconRu{Во́нмем, прему́дрость, во́нмем. Проки́мен глас 4: Помяну́ и́мя Твое́ во вся́ком ро́де и ро́де.}

\ChoirRu{Помяну́ и́мя Твое́ / во вся́ком ро́де и ро́де.}

\DeaconRu{Слы́ши, Дщи, и виждь, и приклони́ у́хо Твое́.}

\ChoirRu{Помяну́ и́мя Твое́ / во вся́ком ро́де и ро́де.}

\DeaconRu{Помяну́ и́мя Твое́}

\ChoirRu{во вся́ком ро́де и ро́де.}

\DeaconRu{Го́споду помо́лимся.}

\PriestRu{Я́ко свят еси́, Бо́же наш, и во святы́х почива́еши, и Тебе́ сла́ву возсыла́ем, Отцу́, и Сы́ну, и Свято́му Ду́ху, ны́не и при́сно, и во ве́ки веко́в.}

\ChoirRu{Ами́нь.}

\DeaconRu{Вся́кое дыха́ние да хва́лит Го́спода.}

\ChoirRu{Вся́кое дыха́ние / да хва́лит Го́спода.}

\DeaconRu{Хвали́те Бо́га во святы́х Его́, хвали́те Его́ в утверже́нии си́лы Его́.}

\ChoirRu{Вся́кое дыха́ние / да хва́лит Го́спода.}

\DeaconRu{Вся́кое дыха́ние}

\ChoirRu{да хва́лит Го́спода.}

\DeaconRu{И о сподо́битися нам слы́шанию свята́го Ева́нгелия Го́спода Бо́га мо́лим.}

\ChoirRu{Господи, помилуй. \emph{(трижды)}}

\DeaconRu{Прему́дрость, про́сти! Услы́шим свята́го Ева́нгелия.}

\PriestRu{Мир всем.}

\ChoirRu{И ду́хови твоему́.}

\PriestRu{От Луки́ свята́го Ева́нгелия чте́ние.}

\ChoirRu{Сла́ва Тебе́, Го́споди, сла́ва Тебе́!}

Во дни оны, воставши, Мариам, иде в горняя со тщанием, во град Иудов. И вниде в дом Захариин, и целова Елисавет. И бысть яко услыша Елисавет целование Мариино, взыграся младенец во чреве ея: и исполнися Духа Свята Елисавет. И возопи гласом велиим, и рече: благословена Ты в женах, и благословен плод чрева Твоего. И откуду мне сие, да прииде Мати Господа моего ко мне? Се бо, яко бысть глас целования Твоего во ушию моею, взыграся младенец радощами во чреве моем. И блаженна Веровавшая, яко будет совершение глаголанным Ей от Господа. И рече Мариам: величит душа Моя Господа, и возрадовася дух Мой о Бозе Спасе Моем. Яко призре на смирение Рабы Своея: се бо, отныне ублажат Мя вси роди. Яко сотвори Мне величие Сильный, и свято имя Его. Пребысть же Мариам с нею яко три месяцы, и возвратися в дом Свой.

\ChoirRu{Сла́ва Тебе́, Го́споди, сла́ва Тебе́!}

\begin{framed}

  \begin{center}
    \textbf{Припевы} \emph{(Трижды)}
  \end{center}

\PriestRu{Пресвята́я Богоро́дице, спаси́ нас.}

\ChoirRu{Пресвята́я Богоро́дице, спаси́ нас.}

\PriestRu{Сла́ва Отцу́ и Сы́ну и Свято́му Ду́ху, и ны́не и при́сно и во ве́ки веко́в. Ами́нь.}

\ChoirRu{И ны́не и при́сно и во ве́ки веко́в. Ами́нь.

  Спаси́ от бед рабы́ Твоя́, Богоро́дице, / я́ко вси по Бо́зе к Тебе́ прибегáем, / я́ко к неруши́мей Стене́ и Предстáтельству.

  При́зри благосе́рдием, Всепе́тая Богоро́дице, / на мое́ лю́тое телесе́ озлобле́ние, / и исцели́ души́ моея́ боле́знь.
}

\end{framed}

\PriestRu{Услы́ши ны, Бо́же, Спаси́телю наш, упова́ние всех конце́в земли́ и су́щих в мори дале́че, и ми́лостив, ми́лостив бу́ди, Влады́ко, о гресе́х на́ших, и поми́луй ны. Ми́лостив бо и человеколю́бец Бог еси́, и Тебе́ сла́ву возсыла́ем, Отцу́, и Сы́ну, и Свято́му Ду́ху, ны́не и при́сно, и во ве́ки веко́в.}

\ChoirRu{Ами́нь.}

\DeaconRu{Па́ки и па́ки, прикло́нше коле́на, го́споду помо́лимся.}

\ChoirRu{Пресвята́я Богоро́дице, спаси́ нас.}

\noindent \textbf{Молитва Божией Матери пред иконой Ее Семистрельной}

\PriestRu{О Многоско́рбная Ма́ти Бо́жия, превы́шшая всех дще́рей земли́ по чистоте́ Свое́й и по мно́жеству страда́ний, Тобо́ю на земли́ пренесе́нных! Приими́ многооле́зненныя воздыха́ния на́ша и сохрани́ нас под кро́вом Твоея́ ми́лости, ина́го бо прибе́жища и те́плаго предста́тельства, разве Тебе́, не ве́мы, но, я́ко дерзнове́ние иму́щи ко И́же от Тебе́ Рожде́нному, помози́ и спаси́ ны моли́твами Свои́ми, да непреткнове́нно дости́гнем Ца́рствия Небе́снаго, иде́же со все́ми святы́ми бу́дем воспева́ти хвалу́ в Тро́ице Еди́ному Бо́гу, всегда́, ны́не, и при́сно, и во ве́ки веко́в. Ами́нь.}

\begin{center}
  \textbf{Отпуст}
\end{center}

\DeaconRu{Прему́дрость.}

\PriestRu{Пресвята́я Богоро́дице, спаси́ нас.}

\ChoirRu{Честне́йшую Херуви́м и сла́внейшую без сравне́ния Серафи́м, без истле́ния Бо́га Сло́ва ро́ждшую, су́щую Богоро́дицу Тя велича́ем.}

\PriestRu{Сла́ва Тебе́, Христе́ Бо́же, упова́ние на́ше, сла́ва Тебе́.}

\ChoirRu{Сла́ва Отцу́ и Сы́ну и Свято́му Ду́ху, и ны́не и при́сно и во ве́ки веко́в. Ами́нь. Го́споди, поми́луй. \emph{(Трижды)} Благослови́.}

\PriestRu{Христос, истинный Бог наш, молитвами Пречистыя Своея Матере, преподобных и богоносных отец наших, и всех святых, помилует и спасет нас, яко Благ и Человеколюбец.}

\ChoirRu{Ами́нь.}

\end{document}
