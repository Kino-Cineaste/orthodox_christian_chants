%************************************************
\chapter{The Gospels}\label{ch:the-gospels}
% ************************************************

\section{Introduction.}

The word Gospel (from god and spell) means good message or news. This name designates the first four books of the New Testament which narrate the life and teachings of the incarnate Son of God, our Lord Jesus Christ---all He did to establish a virtuous way of life on earth and to save sinful mankind.

Jesus Christ said to His contemporaries: ``He who has seen Me has seen the Father'' (John 14:9). Until the coming of the Son of God on earth, people perceived God as the almighty Creator, the stern Judge residing in unapproachable glory. Jesus Christ gave us a new understanding of God: that of a close, compassionate, and loving Father. Truly, all of Christ's features, His every word and gesture were permeated with unending compassion. He was as a physician amongst the sick. People, sensing His love for them, were attracted to Him by the thousands. No one ever heard a refusal. Christ ministered to all. He cleansed the conscience of sinners, healed the weak and the blind, comforted the despairing, and exorcised those possessed by demons. And simultaneously, when the circumstances required, He displayed His divine and sovereign power. Then, everything---nature and the death itself---obeyed His almighty Word.

Through this booklet, we will acquaint the reader with the times and circumstances in which the Gospels were written and will introduce selected teachings of our Savior. It is our goal that the reader will delve more deeply into the life and teachings of our Savior, because the more we read the Gospels, the stronger our faith becomes and the more clearly we see the purpose of our earthly existence. Also as we acquire more experience in spiritual matters, we begin to perceive more obviously the nearness of our Savior. Then He truly becomes our Good Shepherd Who leads us toward salvation.

Especially in our time, when we hear and read about so many contradicting and unfounded opinions, it would be wise to make the Gospels our reference book. Indeed, while all other books contain the opinions of mere mortals, the Gospels reveal to us the eternal Truth.

\section{History of the Gospel Texts.}

All the Sacred books of the New Testament were written in the vernacular Greek, an Alexandrian dialect, called koine. This language was spoken, or at least understood, by all the educated inhabitants of the Eastern and Western parts of the Roman Empire. It was the language of all the cultured people of that time. The Evangelists wrote in Greek rather than in Hebrew, in which the books of the Old Testament were written, in order to make the New Testament books accessible to a maximum number of people.

At that time only the capital letters of the Greek alphabet were used in writing, without diacritics, punctuation, or separation between words. Lower case letters appeared only in the ninth century, together with spacing between words. Punctuation marks were introduced only with the invention of the printing press in the 15th century. The present separations of chapters was introduced by Cardinal Hugo in the 13th century, and the separation into verse was done by the Parisian typographer Robert Stephen in the 16th century.

Through its learned bishops and priests, the Church always concerned itself with preserving the texts of the Sacred Books in their original purity. This was especially important before the introduction of the printing press, when the texts were copied by hand, and errors could easily infiltrate the new copies. It is known that several Christian scholars of the 2nd and 3rd centuries such as Origen; Isihi, the Bishop of Egypt; and Lukian the priest of Antioch, labored with great diligence over the amendment of the Biblical texts. With the invention of the printing press, careful attention was given to the reproduction of the Sacred New Testament Books to ensure that they were copied according to the most ancient and reliable manuscripts. During the first quarter of the 16th century there appeared two publications of the New Testament texts in Greek: the ``Complete Book of Writings'' published in Spain, and the edition of Erasmus of Rotterdam in Basel. By the end of the last century, the scholar Tischendorf completed an important critical edition, for which he compared approximately nine hundred manuscripts of the New Testament.

These conscientious critical works, as well as the untiring efforts of the Church, filled and guided by the Holy Spirit, assure us that presently we possess the original and unadulterated Greek text of the Gospels. It may be fairly said that the genuineness of these books rests upon better evidence than that of any other ancient writing.

The Slavic and Russian translations. During the second half of the 9th century, the Sacred Books of the New Testament were translated into Slavonic by Saints Cyril and Methodius, who enlightened the Slavic people. This language, a Bulgaro-Macedonian dialect, was more or less understood by all of the speakers of Slavic dialects and the people living in the region of Thessalonica, the birthplace of the brothers. The oldest manuscript of this Slavonic translation was preserved in Russia under the title of the Ostromirov Gospel since it was written for the Mayor of Novgorod by deacon Gregory Ostromirov circa 1056-57. With time, the Slavonic text was subjected to some Russification. The contemporary Russian translation was made during the first half of the 19th century.

English Translations. Despite the many English translations of the Bible in whole or in part undertaken during the Middle Ages, it is not until the 16th century that the history of the English Bible as we know it actually begins. The New Testament of William Tyndale, published in 1525-26, was translated directly from the Greek original rather than from the Latin translation known as Vulgate, as its predecessors had been.

The Hampton Court conference in 1604 proposed a new translation of the Bible, and 54 translators were invited to undertake the work at Oxford, Cambridge and Westminster. Their translation, dedicated to King James I, was published in 1611 in large folio volumes. This translation, known also as the Authorized Version, has so embedded itself in the religious and literary history of the English-speaking peoples that its secure place has been challenged only by revisions of it, not by replacements for it. Such revisions came in the British Revised version of 1885, followed by the American Standard Version of 1901. This later was more drastically revised by the Revised Standard Version (1946-52).

The Amplified Bible (1954) is a literal translation with multiple expression using associated words to convey the original thought. This version is intended to supplement other translations. The Jerusalem Bible (1966) is a translation form the Hebrew Masoretic text, the Greek Septuagint, the Dead Sea Scrolls, and accepted Greek and Aramaic New Testament texts. In making the New American Bible (1970), a Catholic translation, all the basic texts were consulted, and the work was 26 years in the making. The Living Bible (1971) is a popular paraphrase edition and is the work of a single translator, Kenneth L. Taylor. The New American Standard Bible (1971) was translated by an editorial board of 54 Greek and Hebrew scholars and required nearly 11 years to complete. The New King James Bible (1979-82) is a version in conformity with the thought flow of the 1611 King James Bible. It is based on the Greek text used by Greek speaking churches for many centuries, known presently as the Textus Receptus or Received Text.

There are more than a dozen English Bible translations available today, each with its merits and its weaknesses. Some of them are more literal and, consequently, more difficult to understand; while others are much more readable and understandable, but less accurate. A serious Bible student might want to compare several of these translations in order to get a better understanding of the original text. The great variability among modern Bible versions testifies to the fact that translating is essentially interpreting. In other words, to do a good job, the translator must know both the original and the language being translated into quite well. The translator must understand the subject, and, what is extremely important, grasp the idea the author intended to convey and the sense in which he intended it to be conveyed. And since the ultimate author of Sacred Scripture is the Holy Spirit, the translator needs His illumination and inspiration to correctly convey His message. St. Peter pointed to this requirement when he wrote: ``No prophecy of Scripture is of any private interpretation, for prophecy never came by the will of man, but holy men of God spoke as they were moved by the Holy Spirit'' (2 Pet 1:20-21). And here lies the main problem with of some of the modern Bible translations. The scientists who did them, with all their knowledge of ancient languages and sincere efforts to do the best job, were often far from the Church and hence never understood its teaching. So at the present time, the King James Bible and its more contemporary version, the New King James Bible, although neither is perfect, seem to convey most accurately the original meaning of the Bible as it was always understood by the Church.

\section{Time of Writing.}

The precise time at which each of the books of the New Testament was written cannot be exactly determined. However, there is no doubt that they were written during the second half of the first century. This is evident from the fact that a series of second-century writings---such as the Apologies of the holy Martyr Justin the Philosopher, written in the year 150, the poetical works of the pagan author Celsus, written in the middle of the second century, and especially the epistles of the Bishop-martyr Ignatius Theophorus (the God-Bearer), of Antioch, written circa 107 A.D.---all make numerous references to the books of the New Testament.

The first books of the New Testament were the epistles of the Apostles, brought about by the need to strengthen the faith of the newly founded Christian communities. Soon, however, there also developed a need for a systematic documentation of the earthly life and teachings of the Lord Jesus Christ. No matter how extensively the so called ``contradictory critics'' have tried to undermine belief in the historical authenticity and originality of our Gospels and other Sacred Books, referencing their origin to a much later time (e.g. Bauer and his school), the newest findings in ecclesiastical literature (especially works of the ancient Church's Fathers), give full support to the conviction that all four Gospels were indeed written in the first century.

Through many inferences, one concludes that the Gospel of St. Matthew was written prior to the others and no later than 50-60 years after the birth of Christ. The Gospels of Sts. Mark and Luke were written somewhat later, but in any event, before the destruction of Jerusalem, that is, before 70 A.D. St. John the Theologian wrote his Gospel later than the others, and most probably at the end of the first century---late in life, when he was over 90 years old. Somewhat earlier, he wrote the Apocalypse or the book of Revelation. The Acts of the Apostles was written shortly after the Gospel of St. Luke, and as indicated by its preface, serves as a continuation of the Gospel according to St. Luke.

\section{The Significance of the Gospels.}

All four Gospels harmoniously narrate the life and teachings of Christ the Savior, His miracles, His sufferings on the Cross, His death and burial, His glorious resurrection from death and ascension into heaven. Mutually supplementing and clarifying each other, they represent a single, whole book, without contradictions or variances in essentials and fundamentals.

The mysterious chariot seen by the prophet Ezekiel at the river Chebar (Ezekiel 1:1-28), with four creatures, whose likenesses were reminiscent of a man, a lion, an ox and an eagle, serves as a symbol for the four Gospels. These likenesses, taken separately, became emblems for the evangelists. Beginning with the 5th century, Christian art represents Matthew with a man or angel, Mark with a lion, Luke with an ox, and John with an eagle.

Besides our four Gospels, there were up to 50 other similar writings during the first centuries which referred to themselves also as ``gospels'' and claimed to be of Apostolic origin. The Church designated these as apocryphal---that is, non-credible, repudiated books. These books contain distorted and dubious narratives. Such apocryphal gospels include: ``the first gospel of Jacob,'' ``the story of Joseph the carpenter,'' ``the gospel of Thomas,'' ``the gospel of Nicodemus,'' and others. In these ``gospels'' one finds most of the oldest legends relating the childhood and youth of Jesus Christ.

\section{The Relationship of the Gospels.}

Of the four Gospels, the content of the first three, those of Matthew, Mark and Luke, conform closely to each other in format and content. In this respect, the fourth Gospel, that of John, remains unique, significantly distinguishing itself from the first three. For this reason, it is customary to refer to the first three Gospels as synoptic, stemming from the Greek word syn-opticos meaning, viewed with the same eye, concordant. But while the first three Gospels are similar in format and content, each still remains unique.

The synoptic Gospels narrate almost exclusively the deeds of the Lord Jesus Christ in Galilee, while that of St. John speaks of our Lord's activities in Judea. The synoptics relate mainly the miracles, parables and events of our Lord's life, while St. John's Gospel discusses the deeper meaning of our Lord's life and cites only His most elevated discourses.

The Gospels, for all their variations, do not contain inherent contradictions. In reading attentively, one easily finds clear signs of agreement between the synoptics and St. John. Although St. John narrates very little about the Lord's ministry in Galilee, he is undoubtedly aware of His repeated, lengthy sojourns there. The synoptics relate nothing concerning the early activity of the Lord in Judea and in Jerusalem itself, although hints of these activities frequently occur. Therefore, according to their observations, the Lord had friends in Jerusalem as well as disciples and followers, such as the owner of the rooming house where the Last Supper took place, and also Joseph of Arimathea. Especially important in this respect are the words quoted in the synoptics, ``Jerusalem! Jerusalem! How often did I wish to gather your children.'' This expression clearly suggests multiple visits of the Lord to that city.

The fundamental difference between the synoptics and St. John lies in their documentation of the Lord's speeches. In the synoptics, these speeches are simple and easy to understand, while St. John's are deep, mysterious and often difficult to understand, as if they were not intended for the multitude but for a more select circle of listeners. This is very true. The synoptics present the Lord's speeches as directed to the simple, ignorant people of Galilee. St. John in general, conveys mainly the Lord's speeches as directed to the learned Scribes and the Pharisees, people who were well acquainted with the Law of Moses, and who more or less stood among the upper echelons of that Jewish society. Besides, as we shall see later, the Gospel of St. John had a special goal, perhaps to more fully and deeply reveal the Divinity of Jesus Christ. This theme, of course, is much more difficult to comprehend than the easily understood parables of the synoptics. Even here though, there is no big divergence between the synoptics and St. John. If the synoptics bring out the more human aspect of Christ and St. John predominantly His Divine nature, it does not mean that the synoptics lack the Divine side, or that St. John fails to show the human side. In the synoptics, the Son of Man is also the Son of God, to whom was given all power in heaven and on earth. Equally, the Son of God, according to St. John, is also a true man, Who accepts an invitation to a wedding feast, speaks as a friend with Martha and Mary and weeps at the grave of His friend Lazarus.

Thus, the synoptics and St. John mutually enhance and complement each other, and only in their unity do they reveal Christ's personality as a perfect God and a perfect Man.

The Orthodox teaching has always maintained that while the Holy Scriptures were a result of Divine inspiration given to the writers, imparting to them thoughts and words, the Holy Spirit did not restrain their individual intellects or suppressed their personal attributes. The descent of the Holy Spirit did not stifle the human spirit, but rather cleansed and elevated it above its ordinary limits. Therefore, while representing in themselves a single unity in interpretation of God's truth, the gospels differ from each other in the personal characteristics of each evangelist, distinguishing themselves in structure, style and form of expression. They also differ as a result of the circumstances and conditions under which they were written, as well as in the objectives which each evangelist set for himself.

That is why, for a better explanation and understanding of the gospels, it is essential for us to more closely familiarize ourselves with the personality, character and life of each of the four evangelists and the circumstances during which each gospel was written.



\section{The Gospel According to St. Matthew.}

The Evangelist Matthew, who also bore the name Levi, was one of the twelve apostles of Christ. Until his calling, he was a publican or Roman tax collector, and as such was disliked by his compatriots, the Jews. The Jews scorned and hated the publicans because they served the infidel rulers of their people and created hardships by levying taxes, frequently overcharging in the process.

St. Matthew narrates about his calling in the 9th chapter of his Gospel, referring to himself as ``Matthew,'' while Mark and Luke call him ``Levi.'' It was customary for the Jews to have several names.

Moved deeply by the Lord's mercy in not loathing him, in spite of the scorn from the Jews and especially from their leaders, the scribes and Pharisees, Matthew wholeheartedly accepted Christ's teachings. He profoundly understood the superiority of Christ's message over the paltry opinions and traditions of the Pharisees. They only looked righteous, but were selfish, cruel and despised simple people. That is why Matthew presents in such detail the accusatory speech of the Lord against the scribes, Pharisees and other hypocrites (see the 23rd chapter of his Gospel). One must assume that for the same reason Matthew took so close to his heart the issue of salvation of his own people saturated with false teachings of scribes. That is also why he wrote his Gospel preeminently for the Jews. There is a basis for assuming that originally he wrote his Gospel in Hebrew, and then some time later had it translated into Greek for the rest of the Church.

In writing his Gospel, St. Matthew's main objective was to prove to the Jews that Jesus Christ is precisely that Messiah Whom the Old Testament prophets had predicted, and that the Old Testament Scriptures become clear and assume their wholeness only in the light of Christ's teaching. That is why he begins his Gospel with Christ's genealogy, showing the Jews His descent from David and Abraham. He makes a considerable number of references to the Old Testament (over 100) in order to prove Christ's fulfillment of the Old Testament prophecies. The designation of this first Gospel as ``for the Jews'' can be seen also in the fact that St. Matthew, unlike the other Evangelists, mentions Jewish customs without explaining their reason and meaning. Similarly, he includes several Aramaic words used in Palestine, without explaining their meaning.

In his Gospel, Matthew gave special emphasis to our Lord's kingly relations and activities. His favorite term for designating the rule of the Messiah was the phrase Kingdom of Heaven. Matthew showed that the Messiah inaugurated His kingdom with all authority in Heaven and on earth being given to Him (ch. 28:18). For the present age, to be sure, that Kingdom must be seen in the loyal submission of His people to Him and their obedience to His rule. The Messiah's kingdom is no less real because it is spiritual. Moreover Matthew assured that the King will return in the regeneration and sit on the throne of His glory while ``you [Apostles] who have followed Me will also sit on twelve thrones, judging the twelve tribes of Israel'' (ch. 19:28).

He also warned his readers that the benefits of the Kingdom would extend beyond the limits of the Jewish race. Disciples are to be made during the present age from among all the nations (ch. 28:19), while the Kingdom grows as a mustard tree grows from a tiny seed (ch. 13:31). In the age to come all the nations will be gathered before the King and the faithful ones invited to inherit the Kingdom (ch. 25:32-34). In this Kingdom people will come from the east and the west and the north and the south and sit down with the patriarchs while the unfaithful sons of the Kingdom (the unbelieving Jews) will be cast out (ch. 8:11-12). Matthew certainly had in mind also to impress upon his Jewish brethren that the mission of the King was to save the people of the Kingdom from their sins---therefore the King's name was Jesus which means ``savior'' (ch. 1:21). In order to save His people, the King gave His life as a ransom (ch. 20:28); His blood was poured out for the remission of sins (ch. 26:28). His power to deliver His subjects from their enemy (the Devil) was demonstrated first by vanquishing him in all his temptations (ch. 4:1-11) and second by coming victoriously alive from the dead (ch. 28).

After preaching in Palestine for a long time, St. Matthew traveled to other nations to spread the Gospel, and ended his life with a martyr's death in Ethiopia.

\section{The Gospel According to St. Mark.}

The Evangelist Mark also bore the name ``John.'' He too was a Jew by birth, but did not belong to the Twelve. For this reason, he could not have been a constant listener and travel companion to Christ as St. Matthew had been.

He wrote his Gospel based on conversations with St. Peter and under his guidance. In all probability, he was an eyewitness only to the last days of the Lord's earthly life. Only the Gospel of Mark mentions a youth who, throwing a cloak over his own naked body, followed the Lord when He was taken prisoner in the garden of Gethsemane, but left his cloak and fled naked when the guards grabbed him (Mark 14:51-52). Ancient tradition perceives this youth as St. Mark himself, the author of the second Gospel. His mother, Mary, is mentioned in the book of Acts as one of the women most devoted to Christ. In her home in Jerusalem the faithful gathered for prayer. It is very likely that the upper room where Jesus ate the last Passover with His disciples and instituted the Eucharist (Holy Communion) was in Mark's home.

Mark later traveled with St. Paul on his first missionary journey; the other traveling companion was Barnabas, a maternal uncle to Mark. Mark was with the Apostle Paul in Rome when he wrote the epistle to the Colossians.

Later, apparently, St. Mark became a fellow traveler and collaborator with St. Peter, which is substantiated by the words of Apostle Peter himself in his first Epistle in which he writes: ``She who is in Babylon, elected together with you, greets you, and so does Mark my son'' (1 Peter 5:13). Most likely Babylon was used as another name for Rome). Prior to his departure, St. Paul summons him again and writes to Timothy: ``Take Mark with you, for I need him to serve'' (2 Tim. 4:11). According to ancient tradition St. Peter designated St. Mark the first Bishop of the church in Alexandria where St. Mark ended his life as a martyr.

According to Papias, Bishop of Hieropolis, as well as that of St. Justin the Philosopher and St. Irenaeus of Lyons, St. Mark wrote his Gospel based on discussions with St. Peter. St. Justin refers to it directly as the ``written recollections of Peter.'' Clement of Alexandria claims that the Gospel of St. Mark essentially represents a written version of St. Peter's sermons, which St. Mark documented at the request of Christians living in Rome. The very context of St. Mark's Gospel testifies to the fact that it was designated for gentiles who converted to Christianity. It minimally references the teachings of the Lord Jesus Christ to the Old Testament and even fewer quotations are cited from the Old Testament Scriptures. Additionally, we find Latin words, such as speculator and others. Even the Sermon on the Mount, which serves as an explanation of the superiority of New Testament Law over the Old Testament, is omitted.

Instead, St. Mark's main objective is to present in his Gospel a strong and clear narration of Christ's miracles, emphasizing through them God's heavenly greatness and omnipotence. In his Gospel, Jesus is not ``a descendant of David'' as in that of Matthew, but the Son of God, Lord and Master, Universal King.

\section{The Gospel According to St. Luke.}

The ancient historian, Eusebius of Caesarea, states that St. Luke came from Antioch, and this gives rise to the assumption that St. Luke was a gentile or proselyte---a gentile converted to Judaism. By vocation he was a physician, as seen in the Epistles of St. Paul to the Colossians. Ancient tradition adds to this that St. Luke was also an artist. The contents of his Gospel explaining in detail the Lord's instructions to seventy disciples, lead us to conclude that he was one of the seventy. His unusually dynamic narration of the Lord's appearance to two disciples on their way to Emmaus, where he refers only to Cleopas by name, attest to his being one of the two disciples deemed worthy of the Lord's appearance (Luke 24:13-33).

Also from the Acts of the Apostles, it is evident that with the second journey of the Apostle Paul, St. Luke became his constant collaborator and an almost inseparable fellow traveler. He was with Apostle Paul at the time of Paul's first imprisonment during which the Epistles to the Colossians and Philippians were written. He was also with him during the second imprisonment when the second Epistle to Timothy was written, and which ended with a martyr's death. It is known that after the death of St. Paul, St. Luke preached and died a martyr's death in Achaia (Greece). During the mid-4th century his holy relics and those of St. Andrew the Apostle were transferred to Constantinople.

As is evident from the preface of the third Gospel, St. Luke wrote it at the request of a prominent man, the most excellent Theophilus, who lived in Antioch and for whom he then wrote the Acts of the Apostles, a seeming continuation of the Gospel narratives (Luke 1:3 and Acts 1:1-2). Incidentally, he not only made use of eye witnesses' accounts of the ministry of Christ, but also of already existent writings relating to the Lord's life and teachings. In his own words, he thoroughly scrutinized and compared those writings. Therefore, his Gospel distinguishes itself by its exceptional accuracy in designating times and places of events and strict chronological succession.

The most excellent Theophilus, for whom the third Gospel was written, did not live in Judea nor did he visit Jerusalem; otherwise it would not have been necessary for St. Luke to make geographic clarifications, e.g., that mount Olivet is near Jerusalem, about a Sabbath's walk, etc. On the other hand, it seems that he was familiar with Syracuse, Phrygia, Puteoli in Italy, the Appian Square and the Three Inns in Rome, all of which were mentioned in the book of Acts and for which St. Luke gives no explanations. According to the assertion of Clement of Alexandria (writing at the beginning of the 3rd century), Theophilus was a rich and well-known resident of Antioch (Syria), who professed faith in Christ and whose house served as a church for Antiochian Christians.

St. Luke's Gospel clearly shows the influence of St. Paul with whom St. Luke collaborated and traveled. As the Apostle to the Gentiles, St. Paul tried most of all to disclose the great truth, that Jesus Christ, the Messiah, came to earth not only for the Jews but also for the gentiles and is the Savior of all the world and all people.

In conjunction with this fundamental idea, which is clearly conveyed throughout the entire third Gospel, Jesus Christ's genealogy is traced to the first ancestor of all mankind, Adam, and to God Himself, in order to emphasize His significance for the entire human race (Luke 3:23-38).

Certain passages, such as the mission of Elijah to the widow in the region of Sydon, the curing of Naaman the Syrian (Luke 4:26-27) from leprosy by the prophet Elisha, the parable of the prodigal son, and that of the publican and the Pharisee are found in tight, inner cohesion with particular development of the teaching of Apostle Paul regarding the salvation not only of the Jews, but also of the gentiles, and of man's acquittal before God not by means of the law, but by God's grace, given exclusively through boundless mercy and God's love of mankind. No one had so clearly portrayed God's love for repentant sinners as did St. Luke, placing in his Gospel a collection of parables and events on this subject. In addition to the parables just mentioned, one also remembers the parables of the lost sheep, the lost coin, the good Samaritan, the repentance of the chief of the publicans, Zacchaeus, and other sections, as well as the profound words that ``happiness exists for God's angels in the repentance of one sinner'' (Luke 15:7).

The time and place of the writing of St. Luke's Gospel can be derived through deduction, that it was written prior to the Acts of the Apostles, which seemingly provided a means for the Gospel's continuation (Acts 1:1). The book of Acts ends with a narrative of St. Paul's two year ministry in Rome (Acts 28:30). This took place approximately 63 years after the birth of Christ. Consequently, the Gospel of St. Luke could not have been written later than this, and presumably was written in Rome.

\section{The Gospel According to St. John.}

The Evangelist St. John the Theologian was a beloved disciple of Christ. He was the son of a Galilean fisherman, Zebedee, and Salome. It appears that Zebedee was prosperous since he had workers, and was also a rather prominent member of the Jewish community. His son John was acquainted with the high priest. John's mother, Salome, is mentioned with the ranks of women who served the Lord with their material resources. She traveled with the Lord to Galilee, followed Him to Jerusalem for the last Passover, and participated with the other myrrh-bearing women in obtaining fragrant oils to embalm Christ's body. Legend has it that she was the daughter of Joseph the hoop maker.

St. John was at first a disciple of St. John the Forerunner. After hearing his testimony to Christ as the Lamb of God who took upon Himself the sins of the world, he and Andrew immediately followed Christ (John 1:37-40). St. John later became a steadfast disciple of Christ when, following a miraculous catch of fish on the Sea of Galilee, the Lord Himself summoned him and his brother James. Together with his brother James and Peter, John was worthy of a special closeness to the Lord, finding himself with Him during the most important and triumphant moments of His earthly life. As such, he merited being present at the resurrection of the daughter of Jairus, seeing the Lord's transfiguration on Mount Thabor, listening to the discussion concerning the signs of the Lord's second coming, and witnessing His prayers at Gethsemane. At the Last Supper, he was so close to the Lord that, in his own words, he rested on Jesus' chest (John 13:23-25), and from this stemmed the name bosom friend, later becoming a designation for someone close. Humbly, without calling himself by name, he nevertheless refers to himself in his Gospel as ``the disciple whom Jesus loved.'' The Lord's love for him became apparent when the Lord, while hanging on the cross, committed His Most Blessed Mother to John's care, saying: ``Behold your Mother.''

Ardently loving Jesus Christ, John was full of indignation against those who were hostile towards Him or distanced themselves from Him. That is why he prohibited those who did not follow Christ from casting out demons in the name of Jesus Christ (Luke 9:49), and asked the Lord for permission to cast fire on the inhabitants of a Samaritan village because they did not accept Him when He journeyed through Samaria (Luke 9:54). For this, the Lord named him and his brother James boanerges, which means ``sons of thunder.'' Feeling Christ's love for him but not yet having been enlightened by the grace of the Holy Spirit, he concurs with his mother when she asks on behalf of her sons for the closest place to the Lord in His future Kingdom. In response they receive the prophecy of the forthcoming cup of suffering (Matthew 20:20-23).

After Christ's Ascension, we frequently see St. John with St. Peter. He together with St. Peter and St. James are considered as pillars of the Church in Jerusalem (Galatians 2:9). Following the destruction of Jerusalem, St. John resides and ministers in Ephesus, in Asia Minor. During the reign of Emperor Domitian, he was exiled to the island of Patmos where he wrote the Apocalypse or Revelation (Rev. 1:9-19). Returning to Ephesus from his exile, St. John wrote his Gospel and died a natural death (the only Apostle to do so) at a very old age in approximately 105 A.D., during the reign of Emperor Trajan.

Tradition claims that St. John wrote his Gospel at the request of the Ephesian Christians. They brought him the first three Gospels and asked him to review them and supplement with the Lord's speeches which he had heard. St. John verified the truth of all that was written in the first three Gospels but found that it was necessary to supplement their narratives and to especially expound and clarify the teachings regarding the Divinity of the Lord Jesus Christ, so that with time people would not think of Him as just the Son of Man. This was particularly necessary since by this time, heretics---Ebionites, Gnostics, and the heretic Cerinthus---had emerged and denounced the Divinity of Christ. St. Irenaeus of Lyon wrote about these circumstances around the middle of the 3rd century.

It is clear that the objective of the fourth Gospel was to supplement the narratives of the other three Gospels.	Distinct from the first three Gospels, it was named the Spiritual Gospel.

The Gospel of St. John begins with the exposition of Jesus' Divinity and further contains an entire series of the most spiritually elevating speeches of the Lord, in which are revealed His Divinity and the deepest mysteries of faith. For example, the conversation with Nicodemus about the birth from above with water and Spirit and the mystery of salvation; the discussion with the Samaritan woman regarding living water and of worship of God in spirit and in truth; the discussion on bread descended from heaven and on the mystery of the Eucharist; the discussion about the good shepherd, and especially touching the farewell conversation with the disciples during the Last Supper, and its wonderful conclusion with the so called High-priestly prayer of our Lord. Here we find a whole series of references by the Lord Himself as the true Son of God. For unveiling these most profound truths and mysteries of the Christian faith, St. John received the respected name of Theologian.

The primary purpose of John in writing the Gospel is stated in chapter 20:31: ``These are written that you may believe that Jesus is the Christ, the Son of God; and that believing, you may have life in His name.'' This statement is partly in answer to the teachings of the Gnostics, a heretical group of John's time who posed as Christians, but who included in their teachings some elements of Greek philosophy, some of the teachings of the Jewish philosopher Philo, and elements of those pagan religions known as the mystery cults, as well as some teachings based on the Old Testament. The Gnostics held generally that the God of the universe was so high and holy that it was impossible for Him to create a material world or to have any dealings with persons possessed of material bodies, that there were innumerable intermediary beings or aeons (some superior spiritual beings, similar to angels), one of whom created the world; and another called the Logos or Word of God, was the only channel through whom God could reveal Himself to the world. Some of them said Jesus was the Logos and therefore of an order of life somewhere between God and man. Obviously such teaching would do great harm to true Christianity. John answered these and other wild claims of that sect by affirming: that the Word (Logos) who reveals God is as eternal as God, that He has fellowship with God, that indeed He is of the same essence as God. John affirmed also that He was made flesh (that is took the nature of mankind including a material body) and lived on the earth as Jesus the only begotten Son of God; that life was in Him; and that He was the light which overcame the darkness (just as He overcame death in His resurrection) and that salvation is to be had in consequence of faith in Him rather than by acquiring a system of hidden knowledge. In setting out the purpose of his work, John declared: ``These things are written that people might have faith in Him as the anointed Savior and the true Son of God and that in consequence of this faith they might have life through His name.''

Pure of heart, having devoted himself to the Lord, and loved by Him in return with a special love, St. John penetrated deeply into the mystery of Christian love. No other Apostle unveiled so profoundly and convincingly as he in his Gospel and three Epistles the Christian teaching of the two fundamental commandments of God---of love for God and of love for neighbor; that is why he is also referred to as the Apostle of Love.

Another unique quality of John's Gospel is that, while the first three Evangelists narrate the preaching of the Lord Jesus Christ primarily in Galilee, St. John describes events and preaching in Judea. Through this, one can determine the length of the Lord's public ministry and the duration of His earthly life. Preaching primarily in Galilee, the Lord journeyed to Jerusalem for all major feast days. As evidenced in the Gospel of John, there were three such trips to Jerusalem before Passover. Prior to the fourth Passover of His public ministry, the Lord accepted His death on the Cross. It follows that the Lord's public ministry lasted nearly three and a half years, and that He lived on earth for thirty-three and a half years (as He entered the public ministry on the thirtieth year of His birth, as attested by Luke 3:23).

\section{Conclusion.}

The Lord Jesus Christ came in order to establish the Kingdom of God among men---a virtuous way of life. He taught us to care about this and beseech: ``Thy Kingdom come, Thy will be done on earth as it is in heaven.'' However, He did not wish to impose this Kingdom by artificial or compulsory means. Therefore He avoided any interference in the political administration of the country, but called men toward repentance or change of heart---toward a spiritual rebirth. This in turn would lead to improvement of all phases of community life.

In reading the history of the spread of Christianity, we see that gradually, as people embraced the teaching of the Savior, favorable social and economic changes evolved in their communities. Truly, Christianity facilitated the abolishment of slavery, elevated the position of women, strengthened family unity, formed charitable organizations, and brought to mankind the highest moral and humanitarian principles. We see a total disparity in countries where non-Christian ideas, such as Fascism or ``scientific'' materialism, are propagated. There, instead of the promised earthly paradise, something akin to hell exists, where, rather than honoring God, a created cult honors the political leader.

Since only God knows all the inadequacies and weaknesses of the human race damaged by sin, only He can help man overcome his bad inclinations and resolve personal, family and community problems. Therefore, one must seek in the teaching of the Savior a directive for aspiration and deeds. His teaching places faith in God and love of neighbor as fundamentals of life. It teaches non-covetousness, compassion, humility and meekness. It calls on all to do good and to develop all the abilities given by God. Christ's teaching brings peace and happiness to the soul. It teaches that man was created for eternal bliss in the Heavenly Kingdom and assists him in attaining it. That is why a Christian must, with concentration and a prayerful attitude, constantly read the Gospels, drawing from them heavenly Wisdom.

\section{Selected Teachings of the Savior.}

Charity: ``Come you blessed of My Father, inherit the kingdom\ldots For I was hungry and you gave Me food; I was thirsty and you gave Me drink; I was a stranger and you took Me in; I was naked and you clothed Me; I was sick and you visited Me; I was in prison and you came to Me'' (Matthew 25:34-36; also see Luke 14:12-15, Luke 21:1-4).

Chastity and Marital Fidelity: (Matthew 5:27-32, 19:3-12).

Courage: ``Watch therefore for you do not know when the master of the house is coming---in the evening, at midnight, at the crowing of the rooster, or in the morning---lest coming suddenly, he find you sleeping'' (Mark 13:33-37, see also Luke 11:24-26 and 21:34-36 and Matthew 8:28-33, parable of the unclean spirit).

Faith: ``For God so loved the world that He gave His only begotten Son, that whoever believes in Him should not perish but have life everlasting'' (John 3:16); ``If you can believe, all things are possible to him who believes'' (Mark 9:23); ``Blessed are those who have not seen and yet believe'' (John 20:29, also see Matthew 16:17, Luke 17:5-10, Mark 16:16).

Fasting: ``However this kind does not go out except by prayer and fasting'' (Matthew 17:21; see also Mark 2:19-22, Matthew 6:16-18, Mark 9:29).

Good Deeds: ``Whatever you want men to do to you, do also to them.'' (Matthew 7:12); ``Let your light so shine before men, that they may see your good works and glorify your Father in heaven'' (Matthew 5:16); ``Whoever gives one of these little ones only a cup of cold water in the name of a disciple, assuredly, I say to you, he shall by no means lose his reward'' (Matthew 10:42; also see Luke 19:11-27, Matthew 25:31-46, Luke 10:25-37, parable of the Good Samaritan, also the parable of the barren fig tree, Luke 13:6-9).

The Grace of the Holy Spirit: ``That which is born of the Spirit is spirit\ldots'' (John 3:6); ``Whoever drinks of this water will thirst again, but whoever drinks of the water that I shall give him will never thirst. But the water I shall give him will become in him a fountain of water springing up into everlasting life'' (John 4:13-14); ``If you then being evil know how to give good gifts to your children, how much more will your Heavenly Father give the Holy Spirit to those who ask Him!'' (Luke 11:13); ``The comforter, the Spirit of truth\ldots He will guide us into all truth.'' (John 16:13, also see John 7:37-39 and 14:15-21, also Mark 4:26-29, the parable of the growing seed; Matthew 13:31-32, the parable of the mustard seed; Matthew 25:1-13, the parable of the ten foolish virgins).

Holy Communion: ``Unless you eat the flesh of the Son of Man and drink His blood you have no life in you. Whoever eats My flesh and drinks My blood has eternal life and I will raise him up at the last day'' (John 6:27-58; Luke 22:15-20).

Humility: ``Blessed are the poor in spirit, for theirs is the kingdom of heaven'' (Matthew 5:3); ``Whoever exalts himself will be abased and he who humbles himself will be exalted'' (Luke 14:11); ``Learn from Me for I am gentle and humble in heart, and you will find rest for your souls'' (Matthew 11:29); ``Whoever desires to be great among you, let him be your servant'' (Matthew 20:26; see also Luke 10:21, Luke 18:9-14, Mark 10:42-45, John 13:4-17, Matthew 20:1-16, the parable of the workers in the vineyard).

Love: ``You shall love the Lord your God with all your heart, with all your soul, with all your mind and with all your strength\ldots  you shall love your neighbor as yourself'' (Mark 12:28-34); ``I desire mercy and not sacrifice'' (Matthew 9:13); ``He who has My commandments and keeps them, it is he who loves Me. And he who loves Me will be loved by My Father \ldots and we will come to him and make Our home with him'' (John 14:15-23); ``By this all will know that you are My disciples, if you have love for one another'' (John 13:35); ``Greater love has no one than this, than to lay down one's life for his friends'' (John 15:13, see also Matthew 5:42-48, John 13:34-35).

Non-Judgment: ``Judge not, that you be not judged. For with what judgment you judge, you shall be judged\ldots '' (Matthew 7:1-2)

Narrow Path: ``Enter by the narrow gate; for wide is the gate and broad is the way that leads to destruction, and there are many who go by it. Because narrow is the gate and difficult is the way which leads to life, and there are few who find it'' (Matthew 7:13-14); ``The kingdom of heaven suffers violence and the violent take it by force'' (Matthew 11:12); ``He who does not take his cross and follow me is not worthy of me'' (Matthew 10:38; also see Luke 13:22-30, Mark 8:34-38, Luke 14:25-27, John 12:25-26).

Patience: ``In your patience save your souls'' (Luke 21:19); ``He who stands to the end will be saved'' (Matthew 10:22); ``Bear fruit with patience'' (Luke 8:15); ``\ldots  remember that in your lifetime you received your good things, and likewise Lazarus evil things, but now he is comforted and you are tormented.'' (Luke 16:19-31, the parable of the rich man and Lazarus).

Prayer: ``Ask and it will be given to you; seek and you will find; knock and it will be opened to you'' (Matthew 7:7-11); ``And all things whatever you ask in prayer, believing, you will receive'' (Matthew 21:22); ``God is Spirit, and those who worship Him must worship in spirit and truth'' (John 4:23-24, see also Matthew 6:5-15, Matthew 18:19-20, Mark 11:23, John 16:23-27, Mark 14:38, Luke 11:9-10, Luke 18:1-8, the parable of the unjust judge).

Prudence: ``Take heed that no one deceives you'' (Matthew 24:4, see also Luke 14:28-33, Luke 16:1-13, the parable of the unjust steward).

Purity of Heart: ``Blessed are the pure in heart for they shall see God'' (Matthew 5:8); ``out of the heart proceed evil thoughts \ldots  These are the things which defile a man'' (Matthew 15:19-20); ``\ldots Those with a noble and good heart, who hear the word and retain it'' (Luke 8:15); ``Whoever does not receive the kingdom of God as a little child, will by no means enter it'' (Mark 10:15); ``You are already clean because of the word which I have spoken to you.'' (John 15:3, Mark 7:15-23).

Reconciliation and Forgiveness: ``Forgive us our debts, as we forgive our debtors. For if you forgive men their trespasses, your heavenly Father will also forgive you'' (Matt. 6:14); ``forgive up to seventy seven times seven'' (Matt. 18:22, see also Matt. 5:23-26, Luke 23:34, Matt. 18:23-35; the parable of the unforgiving servant).

Righteousness: ``Blessed are those who hunger and thirst for righteousness for they shall be filled'' (Matthew 5:6); ``Then the righteous shall shine forth as the sun in the kingdom of their Father'' (Matthew 13:43); ``Therefore you shall be perfect, just as your Father in heaven is perfect'' (Matthew 5:48).

Rejoicing in God: ``Rejoice and be exceedingly glad for great is your reward in heaven'' (Matthew 5:12); ``Come to Me all you who labor and are heavy laden and I will give you rest. For My yoke is easy and My burden is light'' (Matthew 11:28-30); ``I give them eternal life and they shall never perish; neither shall anyone snatch them out of My hand'' (John 10:28); ``your joy no one will take from you'' (John 16:22).

Repentance: ``Repent for the kingdom of heaven is at hand!'' (Matthew 3:2); ``For I did not come to call the righteous, but sinners to repentance'' (Matthew 9:13); ``Whoever commits sin is a slave of sin. If the Son makes you free you shall be free indeed'' (John 8:34-37); ``Unless you repent you will all likewise perish'' (like those crushed by the tower in Jerusalem) (Luke 13:3-5; see also Matthew 4:17, John 5:14, Luke 7:47, Matthew 18:11-14, the parable of the lost sheep, Luke 15:11-32, the parable of the Prodigal Son; Luke 18:9-14, the parable of the publican and the Pharisee).

Temptations: ``If your hand makes you sin, cut it off. It is better for you to enter into life maimed, than having two hands, to go into hell.'' (Mark 9:43-49); ``Woe to the world because of offenses! For offenses must come, but woe to that man by whom the offense comes!'' (Matthew 18:7, Luke 17:1-2).

Thankfulness: ``Were there not ten cleansed? but where are the nine? Were there not any found who returned to give glory to God except this foreigner?\ldots arise, go your way. Your faith has made you well!'' (The story of the ten lepers, Luke 17:11-19).

Tongue: ``How can you, being evil, speak good things? For out of the abundance of the heart the mouth speaks. A good man out of the treasure of his heart brings forth good things, and an evil man out of the evil treasure brings forth evil things. But I say to you that for every idle word men may speak they will give account of it in the day of judgment. For by your words you will be justified, and by your words you will be condemned'' (Matthew 12:34-37, Matthew 5:22).

Trust in God: ``Are not five sparrows sold for two copper coins? And not one of them is forgotten before God. But the very hairs on your head are all numbered, Do not fear therefore; you are of more value than many sparrows'' (Luke 12:6-7); ``Let not your heart be troubled; you believe in God, believe also in Me'' (John 14:1); ``The things which are impossible with men are possible with God'' (Luke 18:27); ``For the Son of Man has come to seek and to save that which was lost'' (Luke 19:10).

Truth: ``For this cause I was born, and for this cause I have come into the world, that I should bear witness to the truth. Everyone who is of the truth hears My voice'' (John 18:37; also see Matthew 13:44-46, the parable of the treasure).

Unity: ``There will be one flock and one shepherd'' (John 10:16); ``That they all may be one, as You Father, are in Me, and I in You; that they also may be one in Us, that the world may believe that You sent Me'' (John 17:21); ``For where two or three are gathered together in My name, I am there in the midst of them'' (Matthew 18:20).

Virtues. The development of good qualities was a constant teaching of the Lord Jesus Christ. For example refer to His Sermon on the Mount (Matthew chs. 5-7) and the Beatitudes, in which are traced the path towards total fulfillment (Matthew 5:3-12). The parable of the sower (Matthew 13:3-23); and especially in the parable of the talents (Matthew 25:14-30) state the importance of development and of the abilities which God gave us. The combination of gifts of grace with the development of abilities (talents) comprises the original wealth of man; that is why it is said that ``the Kingdom of God is within you'' (Luke 17:21).

Will of God: ``Thy will be done on earth as it is in Heaven\ldots '' (Matthew 6:10); ``Not everyone who says to me Lord, Lord, shall enter the kingdom of heaven, but he who does the will of My Father in heaven\ldots '' (Matthew 7:21).

Worldly Cares: ``Seek first the kingdom of God and His righteousness, and all these things shall be added to you. Therefore do not worry about tomorrow, for tomorrow will worry about its own things. Sufficient for the day is its own trouble'' (Matthew 6:19-34); ``For what is a man profited if he gains the whole world, and loses his own soul? Or what will a man give in exchange for his soul?'' (Matthew 16:26); ``Children, how hard it is for those who trust in riches to enter the kingdom of God'' (Mark 10:24, see also Luke 10:41-42, Mark 10:17-27, Luke 12:13-21, parable of the rich fool).

\section{A Prayer before reading the Gospels.}

Make to shine in our hearts, O Master who lovest mankind, the incorrupt light of Thy divine knowledge, and open the eyes of our mind to the comprehension of the preaching of Thy Gospel. Implant also in us the fear of Thy blessed commandments that, trampling down all carnal desires, we may pursue a spiritual way of life, both thinking and doing all things well-pleasing unto Thee. For Thou art the enlightenment of our souls and bodies, O Christ God, and unto Thee do we send up glory, together with Thy Father, who is without beginning, and Thine all-holy, and good, and life-creating Spirit, now and ever, and unto ages of ages. Amen.
