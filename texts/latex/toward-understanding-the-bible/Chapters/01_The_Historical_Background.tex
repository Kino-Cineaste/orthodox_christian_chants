%************************************************
\chapter{The Historical Background}\label{ch:the-historical-background}
%************************************************

The people among whom Jesus lived prided themselves on being children of Abraham and of Israel. The fortunes of the patriarchs, the growth of Jacob's family in Egypt into the nation of Israel, the deliverance of this family-nation from Egyptian bondage, the divine care for it during the forty years in the wilderness, the giving of institutions of worship and a code of moral and civil law, the settlement in the land which had been promised to their fathers---these events along with the numerous miracles of those early ages were matters of common knowledge. They were familiar also with the exploits of the judges, the rise of the kingdom and the age of David, the erection of the first temple under the agency of Solomon, the division of the kingdom---permitted because of Solomon's sins, the continued sinfulness of the people and the preaching of the prophets to give warning, the captivity of the northern kingdom into Assyria and later of Judah, the southern kingdom, into Babylon, and the restoration of Judah, when Cyrus, the Persian who had conquered Babylon, permitted Jews under Zerubbabel to return to their homeland, to restore Jerusalem, and to build the second temple. Moreover, many of the reforms under Nehemiah, who was the last Old Testament political leader, persisted as social customs until the days of the New Testament.

Throughout the Old Testament the nation Israel is seen as a chosen people, separated from the rest of the world. Their advanced moral and civil code, the system of sacrifices given to them at Sinai---especially the sin offerings, the greater accountability in which they were held for waywardness from the high standards they had learned, the captivities into which they were permitted to go because of their lapses into idolatry, and withal the way in which the Jewish people were preserved---all these, as they are set forth in the Old Testament, make an important background of history for a study of the life and mission of the Lord Jesus.

But the New Testament is not just a continuation of the Old. A political situation vastly different from that described in the Old Testament Scriptures and a greatly changed social order are observed at the beginning of the New Testament. An intertestamental interval of more than 400 years came between the time of Nehemiah (last writer of Old Testament Scriptures) and that of John the Baptist. Governmentally this era falls into four epochs: the Persian, the Greek, the Maccabaean, and the Roman.

The Persian Epoch. At the time of Nehemiah, the Persians were ruling over the Jews who had resettled in Palestine; and their rule continued until the fall of the Persian empire to Alexander the Great, or at least until the entry of Alexander into Jerusalem in 333 B.C. During this epoch the High Priest began to exercise civil as well as religious functions, and there first appeared the jealousy and the cleavage between the Jews and Samaritans. Also, the scribes, who became influential interpreters and the teachers of the Mosaic law, made their appearance as a distinct class during this epoch.

The Greek Epoch. This era lasted from the conquest of the land by Alexander until the heroic exploits of the Maccabaean family, who achieved political independence for the Jewish community in Palestine. Under Alexander the Jews lived in comparative peace and prosperity. After the conqueror's death in 323 B.C., Judaea was first made a part of the Greek kingdom in Egypt whose capital was Alexandria, and whose kings are known in history as the Ptolemies. Their rule was sometimes tolerant and beneficent and sometimes cruel and tyrannical. The second Ptolemy was interested in the Jewish sacred books and had the Old Testament translated from Hebrew into Greek, which translation is known as the Septuagint. After 125 years under Egypt, the Jewish people and their homeland were seized by the Greek kings of Syria whose capital was Antioch. This epoch was brought to a close in a series of most violent oppressions and persecutions with much bloodshed, inflicted by Antiochus Epiphanes, known as one of the cruelest tyrants in all history and the prototype of the Antichrist. Many thousands of Jews were slain, and other thousands were sold into slavery. The temple at Jerusalem was desecrated and closed, and the Jewish people were forbidden to worship Yahweh or observe their religious customs, but were commanded to offer sacrifices to the Greek gods.

The Maccabaean Epoch. The oppression and persecution inflicted on the Jews by Antiochus could not do otherwise than arouse resentment on the part of the faithful ones and inspire resistance at the first opportunity. In the little town of Modin, Mattathias, an aged priest, dared to refuse to offer a heathen sacrifice, as the king's officers had ordered him, and in open defiance of the tyrannical rulers, struck dead an apostate younger priest who volunteered to officiate. Then Mattathias and his five sons, aided by their fellow-townsmen, turned on the Greek officers and slew them all. Immediately the challenge went forth to all who were zealous for their traditions and the worship of Yahweh to rally to the hills around the brave old priest, with thousands responding. But Mattathias soon succumbed to the hardships of the camp and the infirmities of age; and leadership of the band of patriots passed to Judas, his third son, known in history as Judas Maccabaeus (Judas the Hammerer). Without doubt, Judas was the most illustrious figure in Jewish history between David and Jesus. Against overwhelming odds this praying genius of battle, won five of the most brilliant victories recorded in history. After one of these victories he led an army of rejoicing citizens into Jerusalem (165 B.C.) to reopen the temple, which had been closed for three years, to cleanse it and its furnishings from the defilement which Antiochus had put upon them, and to dedicate it anew to the service of the true God. At last Judas fell in battle with an overwhelming Syrian-Greek horde, but the fight for freedom went on, led by his undaunted brothers. Jonathan, youngest of the five, a shrewd diplomat, having taken the reins of leadership, secured important concessions for his people from a rival claimant to the throne in Antioch who later came into power.

When Jonathan was treacherously assassinated, Simon, the oldest of the sons of Mattathias, took up leadership in the cause. In 144 B.C. he achieved for his oppressed people full freedom from the Greeks, both through his bravery in battle and by his wise statesmanship. As these heroes were of the priestly family, they served in a double capacity---as political rulers, and as high priests in the cleansed and restored temple. At length Simon, like his brother Jonathan, was betrayed and slain along with two of his sons; but a third son, John Hyrcanus, was quick to take the reins of leadership. After successful struggles to establish his power in the face of the Greek sympathizers, this ruler led a series of expeditions against hostile neighboring tribes, particularly the Idumaeans to the south and the Samaritans to the north. Later he suppressed the unfriendly activities of the tribes that lived east of the Jordan. By these operations he extended the boundaries of his country until they embraced all the lands of the Old Testament twelve tribes.

But succeeding generations were not always as unselfish in spirit or as genuinely patriotic as Mattathias and his sons. A son of Hyrcanus assumed the title King of the Jews with royal pomp, at the same time retaining the high priest's office and function. There were family jealousies and murder in the scramble for the throne and the high priesthood, and at times the people were sorely oppressed.

It was about this time that the sects, the Sadducees and the Pharisees, appeared. Fundamentally, the difference between these groups was religious; but during the Maccabaean epoch they became more or less political parties, the Pharisees being the party of the common people and the supporters of the revolution, and the Sadducees the party of the wealthy aristocrats and sympathizers with the Greeks.

The Roman Epoch. For centuries the power of the Romans in the west had been rising. Their victorious armies were subduing kingdoms around the shores of the Mediterranean Sea and far into the interior, bringing them under Roman rule; nor was the little kingdom of the Jews to escape. A quarrel between two brothers for the high priesthood and the Jewish throne was the occasion for the Romans to seize the country and establish their power over it. When Pompey, the Roman general, came into the country, each of the brothers appealed to the invader for aid on his side of the quarrel. Before Pompey rendered a decision, the younger of the two brothers, who was the more aggressive and in many respects the stronger, seized the city of Jerusalem and fortified it against the Romans. After a long and bloody siege the Romans entered the city. They took the ambitious younger brother and his two sons as prisoners, and making Judaea a Roman province, named the older of the brothers, and the more peace-loving, as high priest and ethnarch. This latter appellation was an empty title, for the real ruler of the country was Antipater, a crafty Idumaean chieftain who never lost an opportunity to increase his own power or advance the interests of his family. He was soon given the title procurator, that is, guardian of the country for the Romans.

Upon the death of Antipater by assassination in 43 B.C. his son Herod (known in history as Herod the Great) became the ruler. After six years of bloody war with the last claimant of the Maccabaean throne and with the invading Parthians, Herod was named King of Judaea by the Romans. His reign was marked by insane jealousy and ruthless bloodshed. He did not hesitate to put to death any who opposed him, or who seemed to obstruct or hinder his rule or his purposes. Among those executed were three of his own sons, his favorite wife, Marianne, and her brother, whom he had shortly before appointed high priest. He was the king when Jesus was born; and his action in having all the young boys of Bethlehem put to death, in order to be rid of One who was reportedly born King of the Jews, is well known. Herod was a builder: he rebuilt many of the cities which had been ravished in the wars. Best known of his building projects was the replacing of Zerubbabel's temple, erected five centuries before, with the magnificent structure which was in use during the life of Jesus.

According to the provisions of Herod's will, his kingdom was to be divided among three of his sons: Archelaus was to be king in Judaea and Samaria, Antipas (the Herod who had John the Baptist beheaded) was to be tetrarch in Galilee and Peraea, and one Philip tetrarch in Itruraea and Trachonitis, a region east of the Sea of Galilee. When he died in 4 B.C., the Roman senate confirmed this arrangement, except that Archelaus was named ethnarch of Judaea instead of king.

Archelaus was a weakling, as cruel as his father, but not so efficient as a ruler. After ten years of misrule the Romans found it necessary to remove him; and at the request of many Jews, Judaea was put under a procurator---or governor---sent directly from Rome. Pontius Pilate, who gave the death sentence against Jesus, was the fifth such governor sent to Judaea.

Political Situation During the Ministry of Jesus. In Luke 3:1 there is a general, though not quite complete, statement of the political situation during the active life of Jesus. The territory ruled by Pilate embraced Judaea and Samaria, all of which lay between the Mediterranean Sea and the Jordan River; that ruled by Herod Antipas included Galilee west of the Jordan valley and Peraea east of that valley (spoken of in the Gospels as the region beyond Jordan). The tetrarchy of Philip lay east of the Sea of Galilee and the upper Jordan. Within the territory of Antipas and of Philip was a group of cities, inhabited mostly by Greeks, which were free from rule of the tetrarchs. Originally there were ten of those, joined in a loose league known as Decapolis. All of them were east of the Jordan.

\section{The Religious Background.}

The Jewish people of our Lord's day were intensely religious. Their history, as it appears in the Old Testament had been written altogether from a religious viewpoint. Their most brilliant heroes of the Intertestamental period were priests, who led the revolt against foreign tyrants for religious reasons, and whose loyal followers were religious enthusiasts. In the years of oppression and bloodshed during the administration of Antipater as governor and the reign of Herod as king, many devout Jews, losing all hope that their nation would gain political freedom gave themselves to a study of the prophecies in their Scriptures and fondly indulged the hope of a coming Messiah-King. Their religious life was expressed in a system known today as Judaism which had been developed during the Intertestamental period from the Mosaic law and the prophets and the interpretative comments of the scribes.

\section{Places of Worship.}

In our Lord's day the Jewish people maintained two institutions of worship---the temple and the synagogue. There was one temple located at Jerusalem in which the priests officiated at sacrifices and offerings. But there was a synagogue in which the Scriptures were read and interpreted in every town or village and even in many foreign cities.

The Temple. In the Old Testament, worship was largely by sacrificial offerings and ceremonial rites. There was very little congregational worship---singing public prayer or public reading of the Scriptures; and formal preaching was unheard of. The first central place of worship was the movable tabernacle built in the wilderness under the supervision of Moses about 1497 B.C.; it was followed by the temple of Solomon (1012-586 B.C.); and this in turn by the temple of Zerubbabel which was erected in 516 B.C. and endured until Herod the Great dismantled it in 23 B.C. so that he might erect a new one. In the new structure the temple proper was completed in a year and a half (22 B.C.), and the courtyards eight years later. The entire structure was not finished until A.D. 64, just six years before it was totally destroyed by the Romans.

The exact plan on which it was built is not known for certain, though many reconstructions of it have been drawn from information found in Josephus and in the Talmud. The whole area enclosed by the outside porch was about twenty-six acres. It included a Court of the Gentiles, a Court of the Women, a Court of the Israelites, a Court of the Priests, and the temple building proper. That building was the heart of the whole institution containing the holy place and the most holy place or Holy of Holies as did the tabernacle and the two temples before it.

As one ``went up into the temple'' from any direction he first entered the Court of the Gentiles through a porch supported by marble colonnades which surrounded the entire structure. The porch on the south end which was known as the Royal Porch had four rows of massive columns; those on the other three sides had only two. The colonnade on the east side which was backed by the east wall of the city was known as Solomon's Porch (John 10:23; Acts 3:11; 5:12). The area immediately enclosed by these porches was called the Court of the Gentiles, because non-Jews might enter into it, but could proceed no farther into the temple. Without doubt, it was in the Court of the Gentiles that a market for sacrificial animals had been set up, along with tables for the money-changers, whose operations Jesus drove out on two occasions. Four gates opened into this court from the outside on the west, one on the north, one on the east, and according to most authorities, one on the south.

Within the Court of the Gentiles was the sacred enclosure, entered by nine gates---one on the east and four each on the north and the south. The gate on the east, leading into the Court of the Women, was the Beautiful Gate referred to in Acts 3:2-10. At each of these gates was a stone with a carved inscription warning all Gentiles, on pain of death, not to enter. The eastern part of the sacred enclosure was the Court of the Women, on a level nineteen steps higher than that of the Court of the Gentiles. In this court, of which the area has been estimated at from one to one and three fourths acres, were the treasury and chambers for storing facilities for various temple operations. Into this court both Jewish men and women might come, but it was as near to the altar or the House of God as the women could approach. On the west of the Court of the Women, and on a higher level, was the Court of the Israelites. Before the gate between the two courts within the Court of the Women were fifteen semicircular steps. The Court of the Israelites (men's court) was little more than a corridor surrounding the Court of the Priests, from which it was separated by a low stone wall. In the Court of the Priests, which contained the large Altar of Burnt Offerings and the laver, the ritual of animal sacrifices was conducted.

Within the Court of the Priests, on the peak of Mount Moriah, twelve steps higher than the surrounding court, was the House of God, the temple proper. It had three parts: the porch and the chambers, which together surrounded the other two parts, the Holy Place, entered from the porch, and the Holy of Holies, behind the Holy Place. The walls, including those of the porch, were said to be 150 feet high. The Holy Place contained the table of shewbread, the golden candlestick (or lampstand), and the golden altar of incense, just as the Holy Place of the tabernacle of Moses had done. But there was no Ark of the Covenant in the Holy of Holies; for that article, which had been the most sacred object in the tabernacle and in the temple of Solomon, was probably consumed in the flames that destroyed Solomon's temple in 586 B.C. In the Holy of Holies of Herod's temple (as of Zerubbabel's temple) there was only a flat stone, on which the high priest placed his censer and sprinkled the blood of the sin offering on the annual Day of Atonement, which was the only occasion on which that room was entered. It was separated from the Holy Place by a veil. It was this veil which was torn in two from top to bottom at the time of the death of our Lord.

As a place of worship the temple was designed for sacrificial ceremonies. Only during the feast seasons did people gather in the courts in considerable numbers, and no provision was made in it for congregational worship. Individuals would come to the temple for private prayer whenever they felt the need of it (Luke 18:10), and groups would gather in prearranged places for prayer meetings (Acts 3:1). The teaching done in the temple usually took place when interested crowds would gather around a teacher to ask questions or to hear what he had to say (Luke 20:1).

The Synagogue. Not all of the worship was centered in the temple. During the intertestamental period the synagogue had arisen, a local institution to which the Jewish people of each community came to worship---not with sacrifices, but with prayers and eulogies and the reading and interpreting of the law and the prophets. No mention of the synagogue is found in the Old Testament, but in the time of our Lord there was one in every town in Palestine, and in foreign cities where there were as many as ten Jewish households. The officials of the temple were the priests, but those of the synagogue were the synagogue-ruler, the elders, and the attendant. The synagogue-ruler arranged for the services. He appointed the leader for each service and selected the one who would read the law and the one who would read the prophets and those who would recite the interpretations of these Scriptures. The elders seem to have formed a sort of advisory board to assist the synagogue-ruler. The attendant combined the work of sexton and teacher, and usually executed the decisions of the other officers.

The order of the services seems to have been eulogies, benedictions, reading and interpretation of the law, reading and interpretation of the prophets, sermon, and benediction. The interpretations were stereotyped translations of the Hebrew Scriptures into the current Aramaic; they were usually given by a scribe if one was present. Any man might be called on to read the different portions of the Scriptures or for a sermon or exhortation, or a man might ask for the privilege of preaching. The benediction was usually pronounced by a priest if one was present; if not, by anyone. We learn from Luke 4:16 that the Lord Jesus was accustomed to regular attendance at the Synagogue in Nazareth and could be depended upon to take a part in the worship.

\section{Times of Worship.}

The important times of worship for the Jews of New Testament days were the weekly Sabbath and the annual feasts

The Sabbath. In the days of the Lord Jesus, and indeed throughout the Intertestamental Period, the Jews had great reverence for the Sabbath as a day of worship, particularly worship in the synagogues. Moses had delivered to the people of Israel rather stringent commandments concerning the Sabbath day (Exod. 20:6-11; 31:14-17; 35:2-3; Lev. 23:3; Num. 15:32-36), but the emphasis in these commandments was on resting on the Sabbath day rather than worshipping. In actual practice, it seems that from the settlement in the land of Canaan until the Babylonian Exile people were lax, if not negligent, in observance of the Sabbath. But it is very probable that during the Exile groups would gather on the Sabbath day for Scripture study, Psalm singing, and prayer. After the restoration of the people to their homes in Palestine, the reforms under Nehemiah reemphasized the Sabbath as a day of rest (Neh. 13:15-22); and with the institution of the Synagogue, it came to be a day of worship, also.

Before the time of the Lord Jesus many of the scribes, in emphasizing the law of the Sabbath, had gone to extremes in the matter of burden bearing and laboring on the Sabbath, and had laid down many rigid rules, and then had provided ways of escape from their own rulings by means that were just as foolish. One prominent cause of conflict between Jesus and the synagogue authorities was the disregard on His part for those Sabbath regulations which had been set forth by the scribes, but which were not in the law given through Moses.

\section{The Feasts.}

The Jews of New Testament times observed many religious feasts and fasts. We shall discuss six feasts and one fast, of which four of the feasts and the fast had their beginning in the law of Moses. The other feasts were of later origin.

The Feast of the Passover. (Exod. 12:1-20; Lev. 23:5-8; Num. 28:12-25). This was the oldest of the Jewish feasts, having been inaugurated in Egypt at the time of the Exodus. It celebrated the deliverance from Egyptian bondage. The people were commanded to meet from year to year in the city of the central place of worship (tabernacle or temple) and repeat the activities of the last night in Egypt. After making sure that no leaven was in the house where they were staying, they would kill the lamb on the fourteenth day of the first month (Abib, or Nisan), roast its meat, and as a group eat it that evening with unleavened bread and bitter herbs. By New Testament times the Jews had made changes in the details of observing the feast. The people would eat at ease rather than in haste, signifying that they were no longer in bondage to the Egyptians; they would pass a cup of wine around the table at intervals, and each one would take a sip from it; the sprinkling of the blood on the door posts and lintels seems to have been discontinued; and they would sing from Psalms 113-118 (the Hallel) during and after the meal.

Since the Jewish day ended, and a new day began at sunset, the actual eating was during the early hours of the fifteenth day of the month. The Feast of Unleavened Bread followed the Passover proper and lasted eight days, during which there were special sacred meals and sacrifices. The first day of the feast and the last day were holy convocations, regardless of what day of the week they fell on. Sometimes the whole occasion was spoken of as the Passover. The time of the year was March-April. Since Jesus was crucified at the time of the Passover and was raised from the dead on the third day thereafter, the Jewish Passover and the Christian Pascha, which celebrates the resurrection, come at the same season of the year.

The Feast of Pentecost (Lev. 23:15-20 Num. 28:26-31). This was a feast of the first fruits of grain, coming fifty days after the Passover. It was a thanksgiving for the crops ready for harvest, and a presentation of the first fruits of the harvest to the Lord and to His priests. It is sometimes called the Feast of Weeks, because it came seven weeks---a week of weeks---after the Passover. The celebration was at the tabernacle or temple and lasted only one day. That day also was the anniversary of the giving of the law (the Ten Commandments) by the Lord God from Mount Sinai. To Christians it is familiar because on the day of this feast the Holy Spirit came with power upon the group of disciples who were the nucleus of the early Jerusalem church (Acts 2:1).

Feast of Trumpets (Lev. 23:23-25; Num. 29:1-6). Every time this occasion is mentioned in the Bible, it is said to be the first day of the seventh month, but it has long been observed by the Jewish people as New Year's Day (Rosh Hashanah, Head of the Year). Probably even before the time of the Exodus from Egypt it had been celebrated as the beginning of the crop year, because it came after the harvest of the previous year's crops and before the sowing of the crops for the coming year. According to their civil calendar it was the beginning of the year, but according to their religious calendar it was the beginning of the second half of the year. It was a one-day feast observed at home.

The Day of Atonement (Lev. 16:1-34; 23:26-32; Num. 29:7-11). This day, probably the most sacred of the year for a devout Jew, was observed the tenth day of the seventh month. The people remained at home, abstaining from food throughout the day (presumably occupied in confession, repentance, and prayer) while the high priest offered sin offerings to make atonement for the sins committed by the people during the past year. It was the only day during the year when he went into the Holy of Holies taking the blood of the sin offering.

The Feast of Tabernacles (Exod. 23:16; Lev. 23:34-44; Num. 29:12-40; Deut. 16:13-15 cf. Neh. 8:13-18). This was an eight-day feast beginning the fifteenth day of the seventh month in the religious calendar. Thus the people generally had just enough time to go from their homes to the tabernacle or temple after the Day of Atonement. Its purpose was probably twofold. It was a thanksgiving for the crops already gathered. It was therefore sometimes called the Feast of Ingathering (Exod. 23:16; 34:22). To this feast they would take the tithes of the previous year's harvest and increase of cattle. It also celebrated God's care for the Israelites during the forty years of wandering in the desert. Three practices engaged in during the week commemorated the providential care for their fathers. During the week the people dwelt in booths in imitation of their fathers dwelling in tents in the wilderness (Lev. 23:40-43; Neh. 8:14-15). Great candelabra with many lights were erected in the Court of the Women in commemoration of the pillar of fire which guided the people in the wilderness by night. On the last day of the feast a pitcher of water was brought from the pool of Siloam by the multitude and poured out with great ceremony at the foot of the altar in the Court of the Priests in commemoration of the water which the Israelites had received from the Lord out of the rock (Exod. 17:5-6; Num. 20:11). John has given an account of one Feast of Tabernacles which Jesus attended (ch. 7).

\section{The Scriptures.}

The Jews of New Testament times, including Jesus, regarded the Old Testament as the word of God (John 10:35). At that time they had come to think of their Scripture as composed of three groups of books: the Law, the five books of Moses; the Prophets, including many books of history as well as most of the books of prophecy; and the Writings, including the Psalms and many other books of our Old Testament (Luke 24:44). In their minds the books of the Law came from God through Moses (John 7:19, 9:28-29). Moses was insistent that the commandments and the other things he wrote should be received and kept as coming from God (Deut. 6:6; 31:9-13, 24-26) and from the time of the settlement in the land of Caanan these books of Moses were regarded as God's law (Josh. 1:8; 8:32-36). There were, however, long periods of neglect of the law. At the time of the captivity the Jews must have been permitted to take with them to Babylon copies of the law and of other treasured books---history and prophecy and the Psalms and books of wisdom. A new interest in the study of the law was stirred during the Babylonian Exile. At that time the Jewish captives, being in a strange land and deprived of their temple and their sacrificial system would gather in groups for a study of the law, the singing of the Psalms and prayer (Ezek. 8:1; Ps. 137).

Ezra who lived first at Babylon and then at Jerusalem shortly before the close of the Old Testament period, is credited with bringing together the books of the Old Testament. He was of the priestly family and he also designated himself ``a ready scribe'' (Ezra 7:1-6, 12). When he migrated to Jerusalem he aroused a lively interest in studying the sacred books so that from his days the Scriptures were the principal influence among the Jewish people.

The Old Testament was originally written in Hebrew except for small portions of Jeremiah, Daniel and Ezra which were written in Aramaic---a language closely resembling Hebrew. About 250 B.C. a translation into Greek was made at Alexandria (Egypt) known as the Septuagint because the work was done by seventy scholars. That translation was made from a Hebrew text which differed slightly at many places from the text accepted by the scribes (the Masoretic text), but the Septuagint was very influential in New Testament times. In making quotations from the Old Testament, Jesus and the apostles would sometimes quote from the Hebrew and sometimes from the Septuagint, and that fact accounts for some differences between New Testament quotations from the Old Testament and the way those passages read in our Old Testament.

Many Jews in the days of Jesus had come to give to traditional interpretations of the law by the scribes equal weight of authority with the law itself. These are referred to in Matt. 15:2 and Mark 7:5 as the tradition of the elders. This tradition was gathered together in the third century A.D. in a work known as the Mishnah. By the end of the fourth century it had been enlarged with much other material into a voluminous work known as the Talmud, which has been authoritative for Jewish rabbis down to the present.

The fourteen books which we know as the Apocrypha or Deuterocanonical were in existence in the days of Jesus. First Maccabees probably sets forth authentic history and portrays inspiring examples of courageous loyalty to true religion; but the other books of this collection are of little historical or religious value. It is possible that they influenced to some extent the thought of the people of New Testament times. The early Christians, though they permitted these books to be read for edification, considered them not so important as the Canonical books.

\section{Religious Sects and Classes of People.}

Some of the influential groups or parties of the people mentioned in the Gospels were the priests, the scribes, the Pharisees, the Sadducees, the Herodians, the publicans, and the Samaritans. Besides these, there were the Essenes and other similar groups, who are not mentioned in the Bible, but who are thought by some Bible scholars to have been influential among the people during New Testament times.

The Priests. At the beginning of the history of Israel as a nation Aaron, the brother of Moses, of the tribe of Levi, was named high priest, and his sons were named priests with him. After that the priesthood and the high priesthood were hereditary in the family of Aaron. There came to be so many priests, that in the days of David they were grouped into twenty-four courses (I Chron. 23:1-10). A priest without special favor might serve in the temple only a few times in his life, and many who were of the priestly family never had an opportunity to serve. Apart from their function in the temple, there was doubtless an honor and dignity attached to priestly lineage. In Old Testament times one consecrated as high priest would normally serve for life but during the Intertestamental Period the foreign powers exercising rule began to claim the prerogative of appointing the high priest---of course confining the appointments to members of the priestly families. During the Maccabean epoch the high priest had significant political power; and after the Romans seized the country the high priest continued to exercise great influence, being ex-officio president of the Sanhedrin. Consequently the Roman rulers took to themselves the authority to appoint and to remove high priests. A high priest might (and many did) lose favor with a Roman ruler so that he would be displaced by another after serving only a short time.

Mention is frequently made in the Gospels of chief priests who were members of the Sanhedrin. The Sanhedrin was comprised of the high priest at the time, any person who had previously occupied that office, and probably also the heads of the twenty-four courses of priests.

The Scribes. The scribes as a class probably first appeared during the Babylonian Exile (Ezra 7:6). At first they were professional writers who made copies of the law for those who desired them. Since they would naturally soon know more about the law than anyone else they came in time to be teachers of the law and its interpreters. From among their number came the lawyers and the professional rabbis. The most learned among them were doctors of the law. The tradition of elders which was so highly regarded by the Pharisees was composed largely if not altogether of the interpretations of the law which the learned scribes had made.

The Pharisees. This group was doubtless the most influential of the religious sects of the time of Jesus. The roots of some of their practices may be seen in the reforms and the prayers of Nehemiah (Neh. 13:14), but they had their beginning as a group with the struggles against the paganizing Greeks in the days of Mattathias and Judas Maccabaeus. At first they were called Chasidim (Separatists) because of their determination to keep themselves (and the nation as much as possible) from contaminating foreign influences. During the time of Jesus their distinguishing characteristic was the great emphasis they put on keeping the law. By their selfish credit-seeking conformity to legal requirements they sought to bring God under obligation to themselves. They regarded the interpretations by the scribes (the tradition of the elders) as equally authoritative with the written Law itself. They looked on themselves as righteous (and sometimes were so regarded by their fellows) and were highly critical of others. Those who disregarded their rules and standards were called sinners. They believed in the existence of angels, in life after death, and in a future resurrection of the unjust and the just. In general, they were the conservative element of Judaism.

The Sadducees. The Sadducees were opposed to the Pharisees. For the most part, they were priests who were willing to compromise their Jewish principles for favors from the foreign rulers. Probably they began to appear as a separate class during the closing years of the Greek period. They took their name from Zadok, the priest who was faithful to David and Solomon when Abiather, the other priest, fell away to Adonijah (I Kings 1:32-34). Their distinguishing doctrines and characteristics were: they denied the existence of angels, the immortality of the soul, and any idea of a future resurrection. They rejected the ``tradition of the elders'' and the so-called oral law, accepting as authoritative only the written Old Testament. They were severe in their judgment, and were not very popular with the common people.

The Publicans. When the Romans conquered Judaea and made it a part of the Empire, they imposed Roman taxes on the people. The publicans were Jews who collected those taxes for the Romans. Ordinarily, tax collecting was a lucrative employment, because the collectors paid a stipulated amount to the Romans and took from the people what they saw fit, or what they could. But the publicans were hated by the people generally because they were collecting the taxes for the foreign conquerors, and they frequently extorted from the people more than was due, and consequently were rich. Of course, the publicans did not pretend to keep the Jewish law with any degree of exactness, and they were usually classed with the sinners. Jesus was called a friend of the publicans because He was willing to receive those that came to Him, and to accept the hospitality of those who invited Him into their homes; but of course He did not condone their extortion.

The Samaritans were a mixed race. They were descended from those Israelites of the northern kingdom who were left in the country when northern Israel was taken captive by the Assyrians, and of the foreigners that came to live around the city of Samaria. They worshipped Yahweh, but in their worship they mingled many heathen ideas. During the Persian epoch they built a temple on Mt. Gerazim, in which their priests officiated for about 275 years. This temple was destroyed by John Hyrcanus (121 B.C.) and was never rebuilt; but the Samaritans continued to worship on and around Mt. Gerazim. The Jews despised them because of the impurity of their race and because of the readiness with which they made religious compromises with the Greeks and other foreigners. They are still in existence today, but their number has dwindled to a few hundred. They possess a very ancient manuscript of the books of Moses, which is of great value in the study of the Old Testament.

The Essenes. Josephus, the historian, and Philo, the philosopher, tell in their writings of a Jewish sect known as Essenes, who lived during the first century. These people are not mentioned in the Bible. Some of them lived in groups or quarters to themselves in many of the cities and villages of the land, but those about whom most has been written lived like monastics, withdrawn from the world on the west shore of the Dead Sea, supposedly near the town of Engedi. In some matters the teachings of these people resembled those of the Pharisees, but they renounced worldly wealth and followed a rigid schedule of holy living. They did not practice animal sacrifices but sent other gifts to the temple in Jerusalem. The greater part of them renounced marriage and all activities for pleasure. New members were put through three years of rigorous trial during which at intervals certain secret knowledge was imparted to them. In fact, in some respects, they resembled a secret order.

Interest in this group has been revived by the discovery of the Dead Sea Scrolls in 1947 and later, which has brought to light a similar group that lived at Qumran, a considerable distance north of Engedi but also near the Dead Sea. Some scholars maintain that these were the Essenes, and that Qumran was their place of residence rather than Engedi. But the customs and teachings of that group differ considerably from those related by Josephus and Philo. Some scholars have held that John the Baptist came under the influence of these groups. But evidence for such a view is far from convincing---indeed, to me, some of the conclusions suggested seem to be arbitrary and unrelated to the evidence offered.

\section{The Sanhedrin.}

The word Sanhedrin is not found in our English New Testament, but in the Greek original repeated mention is made of a council or court composed of the chief priests, elders, and scribes. This council is not spoken of in the Old Testament; it probably arose during the Intertestamental Period, possibly in the Maccabaean epoch. It included seventy-one members chosen from the three groups most influential with the people. The high priest was always one of its two presidents. The place of its meeting is not certain; the Talmud indicates that it was the Hall of Hewn Stone in the temple, but Josephus mentions the meeting place as outside the temple. The membership included both Sadducees and Pharisees.

During New Testament times it had authority in religious and most civil matters, and limited authority in criminal matters. In cases involving capital punishment, approval of the Roman procurator, or governor, was required before execution of the sentence. Regularly there were no sessions at night, or on a Sabbath day. A sentence of capital punishment could not be passed on the day of the trial. The decision of the judges had to be examined on the following day.

\section{The Messianic Hope.}

Many prophecies about the coming of the Messiah or Christ are found in the Old Testament. Sometimes the prophecies are dim, but understandable to us when we view them through the New Testament records as glasses, as in Gen. 3:15, where it is merely stated that the Seed of the woman would bruise the head of the serpent while the serpent would merely bruise the heel of the promised Seed; but in many of them the promise is clear and assuring. A great many of the prophecies promise a glorious kingdom presided over by a God-sent King who would deliver His people from their enemies and reign in righteousness; and repeatedly it was foretold that this king would be of the lineage of David (Ps. 89:3-4; Isa. 11:1-10; Jer. 23:5-6). The Jews of the Intertestamental Period, suffering from misrule and oppression of the Greeks and the Romans, found comfort and inspiration in anticipation of the promised King and His Kingdom; and many of the scribes gave themselves to a careful study of those prophecies. As a result of that study some of the scribes had outlined a program of the messianic age. From Matthew 16:14 and John 1:21, it is evident that their program included the appearance of an Old Testament prophet, the reappearance of Elijah, and the appearance of the Messiah.

Before the time of Jesus false messiahs had arisen (Acts 5:36-37), who, while enticing multitudes of followers, came to disastrous ends. The scribes could readily inform Herod that the Christ would be born at Bethlehem (Matt. 2:5-6), and without hesitation they answered Jesus that the Christ would be a descendant of David (Matt. 22:42). At the time of the ministry of John the Baptist the people were in expectation (Luke 3:15), which was shared even by the Samaritans (John 4:25). There were many pious ones who were waiting for the redemption of God's people; notably Zacharias, father of John the Baptist, Simeon and Anna (Luke 2:25-38), and Joseph of Arimathaea (Luke 23:50-52).

Besides those prophecies which promised a royal Messiah, there are others (Ps. 22:1-21; Isa. 53, and others) which portray a suffering One, who would bear the sins of the people. Christians confidently identify that One as the Christ, who was promised to be both King and Savior. It is noteworthy also that Ps. 22:22-31, Ps. 110:2-3, and Isa. 53:10 suggest a spiritual Kingdom, in contrast with the glorious visible Kingdom pictured in the other prophecies. But the self-righteous Pharisees and the politically-minded Sadducees did not recognize Jesus as the fulfillment of their cherished messianic prophecies. Accordingly, they condemned Him to death, and thus they unknowingly fulfilled prophecies about the Messiah. But Christians recognized Jesus of Nazareth to be the fulfillment of all messianic prophecies---they trusted Him as Savior, they acknowledged Him as spiritual Lord and King in their lives.

\section{The youth of Christianity.}

Christianity is relatively young. Compared with the course of mankind on the earth, it began only a few moments ago. No one knows how old man is. That is because we cannot tell precisely when a creature which can safely be described as human first appeared. One estimate places the earliest presence of what may be called man about 1,200,000 years in the past. A being with a brain about the size of modern man may have lived approximately 500,000 years ago. In contrast with these vast reaches of time the less than two thousand years which Christianity has thus far had are very brief. If one accepts the perspective set forth in the New Testament that in Christ is the secret of God's plan for the entire creation, and that God purposes to ``gather together in one all things in Christ, both which are in heaven and which are on earth,'' Christianity becomes relatively even more recent, for the few centuries since the coming of Christ are only an infinitesimal fraction of the time which has elapsed since the earth, not to speak of the vast universe, came into being.

When placed in the setting of human civilization Christianity is still youthful. Civilization is now regarded as having begun from ten to twelve thousand years ago, during the last retreat of the continental ice sheets. This means that Christianity has been present during only a fifth or a sixth of the brief span of civilized mankind.

Moreover, Christianity appeared late in the religious development of mankind. It may be something of this kind which was meant by Paul when he declared that ``in the fullness of time God sent forth His son.'' We need not here take the space to sketch the main outlines of the history of religion. We must note, however, that of those faiths which have had an extensive and enduring geographic spread, Christianity is next to the latest to come to birth. Animism in one or another of its many forms seems to have antedated civilization. Polytheisms have been numerous, and some of them, mostly now merely a memory, are very ancient. Hinduism in its earlier aspects antedates Christianity by more than a thousand years. Judaism, out of which Christianity sprang, is many hundreds of years older than the latter. Confucius, the dominant figure in the system which the Occident calls by his name, lived in the sixth and fifth centuries before Christ. The years of the founder of Buddhism, although debated, are commonly placed in the same centuries. Zarathustra, or, to give him the name by which English readers generally know him, Zoroaster, the major creator of the faith which was long official in Persia and which is still represented by the Parsees, is of much less certain date, but he seems to have been at least as old as Confucius and the Buddha and he may have been older by several centuries. Only Manichæism and Islam were of later origin than Christianity. Of these two, Manichæism has perished, Christianity is, therefore, the next to the youngest of the great religious systems extant in our day which have expanded widely among mankind.

That Christianity emerged in the midst of a period in which the major high religions of mankind were appearing gives food for thought. Most of these faiths came into being in the thirteen centuries between 650 B.C. and A.D. 650. Of those which survive only Judaism and Hinduism began before 650 B.C. Here was a religious ferment among civilized peoples which within a comparatively brief span issued in most of the main advanced religions which have since shaped the human race. This occurred with but little interaction of one upon another. Only Christianity and Islam are exceptions. Both of these were deeply indebted to Judaism, and Islam was influenced by both Judaism and Christianity.

The youth of Christianity may be highly important. It might conceivably mean that, as a relatively late phenomenon, Christianity will be transient. The other major religions have risen, flourished, reached their apex, and then have either entered upon a slow decline or have become stationary. Hinduism is not as widely extended as it was fifteen hundred years ago. Not for five centuries have important gains been registered by Buddhism and during that time serious losses have occurred. Confucianism has achieved no great geographic advance since it moved into Annam, Korea, and Japan many centuries ago, and at present it is disintegrating. Islam has suffered no significant surrender of territory since the reconversion of the Iberian Peninsula to Christianity, a process completed about four centuries ago, and in the present century has pushed its frontiers forward in some areas, notably in Africa south of the Sahara. Yet its advances have been much less marked than in the initial stages of its spread. It might be argued that Christianity is to have a similar fate and the fact of its youth may mean that for it the cycle of growth, maturity, and decay has not reached as advanced a stage as has that of other faiths. To this appraisal the fact of the emergence of the high religions, including Christianity, in the comparatively brief span of thirteen centuries may lend support. The grouping of their origins in one segment of time and the progressive weakening of so many of them might be interpreted as an indication that all religions, in the traditionally accepted use of that term, and including even Christianity, are a waning force in the life of mankind. Some, indeed, so interpret history and declare that the race is outgrowing religion. The losses in Europe in the present century might well appear to foreshadow the demise of Christianity.

On the other hand, the brief course of Christianity to date may be but a precursor to an indefinitely expanding future. The faith may be not far from the beginning of its history and only in the early stages of a growing influence upon mankind. As we are to see more extensively in subsequent chapters, the record of Christianity yields evidence which can be adduced in support of this view. As we hinted in the preface and will elaborate more at length later, the faith has displayed its greatest geographic extension in the past century and a half. As the twentieth century advances, and in spite of many adversaries and severe losses, it has become more deeply rooted among more peoples than it or any other faith has ever before been. It is also more widely influential in the affairs of men than any other religious system which mankind has known, The weight of evidence appears to be on the side of those who maintain that Christianity is still only in the first flush of its history and that it is to have a growing place in the life of mankind. In this Christianity is in striking contrast with other religions. Here are much of its uniqueness and a possible due to its significance.

\section{The limited area of early Christianity.}

The cultural area in which Christianity arose, that of the Mediterranean Basin, was merely one of the centres of contemporary civilization and embraced only a minority of mankind. It is important that this fact be recognized if we are to see the history of the faith in its true perspective. Since during the past four and a half centuries the Occident and its culture have been progressively dominant throughout the globe, and since in connection with it Christianity has had its world-wide spread, we are inclined to regard that condition as normal. In view of the circumstance that during its first five centuries Christianity won the professed allegiance of the Roman Empire, which then embraced the Occident, many have thought of it as having at this early date conquered the world. This is entirely mistaken. East of the Roman Empire was the Persian Empire which for centuries fought Rome to a stalemate. Its rulers regarded Christianity with hostile eye, partly because of its association with their chronic rival, and fought its entrance into their domains. India, although not united into one political realm, was the seat of a great culture which influenced the Mediterranean area but which, in spite of extensive commercial contacts, was but little affected religiously by the Occident. China had a civilization all its own. At the time when the Roman Empire was being formed, China was being welded into a political and cultural whole under the Ch'in and the Han dynasty. In area it was about as large as the Roman Empire. In wealth and population it may not have equaled its great Western contemporary, but in cultural achievements it needed to make no apology to India, Persia, or Rome. In the Americas were small beginnings of civilized states-In its first five centuries neither China nor America was reached by Christianity. These civilizations, even when taken together, occupied only a minority of the surface of the earth. Outside them were the vast masses of ``primitive'' mankind, almost untouched by Christianity until after its first five centuries were passed. It is against this background that we must see the rise and early development of Christianity. In its initial centuries the geographic scope of Christianity was distinctly limited.

\section{The unpromising rootage of Christianity.}

When we come to the area in which Christianity began, we must remind ourselves that even there, in that geographically circumscribed region, the roots from which it sprang appeared to promise no very great future for the faith. It is one of the commonplaces of our story that Christianity was an outgrowth of the religion of Israel. Israel was never important politically. Only for a brief time, under David and Solomon, between nine hundred and a thousand years before Christ, did it achieve a domain of considerable dimensions. Even then it did not rank with the major empires. That realm soon broke up into two small states, the Northern and Southern Kingdoms, insignificant pawns in the contests among the great powers in the valleys of the Nile and of the Tigris and Euphrates Rivers. Except for what came through its religion, Israel was of slight consequence culturally. When contrasted with its neighbours in Mesopotamia and Egypt it occupied a small and infertile area in the Palestinian uplands. Its cities were diminutive and its buildings unimpressive. Its art was not distinguished. Today the monumental ruins of Egypt, Nineveh, Babylon, and even Syria dwarf those of Israel's past and make clear the relative insignificance, from the political and economic standpoint, of the land in which was the stock from which Christianity sprang.

In this respect Christianity was in striking contrast with those faiths which became its chief rivals. The polytheisms which it displaced in the Mediterranean Basin had the support of old and politically powerful cultures and states. Zoroastrianism was associated with Persia, for centuries one of the mightiest empires on the globe. Hinduism was indigenous to India, one of the major cultural centres of mankind. Buddhism was also a native of India and early won wide popularity in the land of its birch. Both Hinduism and Buddhism owed much of their extension outside India to the commerce and the cultural prestige of that land. Confucianism was for two thousand years so closely integrated with China, one of earth's mightiest civilizations, that its spread on the periphery of that realm seemed assured. Islam early brought unity to the Arabs and within a century of its origin was supported by one of the three largest and strongest empires of the day. At its outset Christianity had no such potent associations to commend it. Not until, after more than three centuries, it had, through its first amazing victories, become dominant in the Roman Empire did it achieve such an influential cultural and political alliance as these other faiths early possessed.

It is sometimes said that Israel owed its unique religious development to the fact that it was of the family of Semitic peoples and was on the land bridge between the great civilizations of Egypt and of Mesopotamia and so was stimulated by contributions from each of them. But there were other Semitic peoples who were in much the same favored position, the Phoenicians among them, and it was only in Israel that the religious development occurred which issued in Judaism and Christianity.

Moreover, it was in a minority, even within the comparatively obscure people of Israel, that the stream which issued in Christianity had its rise and its early course. The prophetic monotheism which was the source of Christianity long commanded the undivided support of only a small proportion of Israel. The loyal minority were sufficiently numerous to cherish and hand down the writings of the prophets. Through them came the main contributions of Israel to the world. Within this minority we find the direct antecedents of Christianity. Yet the majority of Israel either rejected the prophets outright or devitalized their message by compromise. Even among the relatively insignificant people within which Christianity arose, only the numerically lesser part could be counted in the spiritual ancestry of the faith. Fully as significantly, it was largely those who believed themselves to be in the succession of that minority who so opposed Jesus that they brought him to the cross.

Christians have seen in this story the fashion in which God works. They have believed that always and everywhere God has been seeking man and has been confronting man with Himself and with the standard which He has set for man. Yet man, so they have held, persistently rebels against God and becomes corrupt. God, of His mercy and love, has wrought for man's redemption. This He has not done in the way which men would have predicted. Even those whom men have accounted wise have been so blinded by sin, especially by pride and self-confidence, that they could not clearly see or hear God. For reasons known only to Himself, so Christians have maintained, God chose as His channel for man's salvation a small, insignificant minority among the people of Israel, themselves of slight consequence in physical might. As the culmination of His revelation of Himself and His redemption of man, He sent His son, who, the heir of this humble minority and building on the foundations laid by them, became the centre of the Christian faith.

The story, as seen from a Christian standpoint, might also be put in the following fashion. God has always, from the beginning of the human race, been seeking to bring men into fellowship with Himself and into His likeness. He has respected man's free will and has not forced Himself on man. Only thus could He produce beings who are not automata, but are akin to Himself. In response to God's initiative, men everywhere were stimulated to grope for God. As a result of their seeking, various religions arose. All of these, clouded by man's sin, were imperfect and could not meet man's need or fulfil God's purpose. For some inscrutable reason, God found among the people of Israel a minority who responded to Him and, therefore, was able to disclose Himself fully through one who came out of that succession and through him made possible the salvation of man.
